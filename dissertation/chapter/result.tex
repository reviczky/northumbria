\startcomponent result
\product dissertation

\usemodule[filter]
\defineexternalfilter[R]
  [
    filtercommand={R CMD BATCH -q --save --restore \externalfilterinputfile\space \externalfilteroutputfile},
    output=\externalfilterbasefile.out,
    directory=stan,
    readcommand=\typefile,
    read=no,
    cache=yes,
  ]

\setuphead[chapter][textstyle={\tfd\sc}]
\startchapter[title={Results Collected and Analysis},reference=chapter:result]

% This section should assess your results/findings and your analyses of these.
% Examples of techniques you may use include reasoned evaluation, thorough product testing, scientific testing, data analysis, internal and external validation and statistical survey.
% You would normally make use of more than one technique, and your use of the techniques should reflect the postgraduate nature of the research.
% You should also justify your choice of techniques.
% This section also assesses your discussion on how your findings contribute to the wider academic body of knowledge, and any comparison of your results/findings/hypotheses with those of others.

Assessing the results collected and findings from the work done in the previous chapter, the reasoned evaluation and data analysis will be discussed here, that will be closed off with the validation of the data. \par
Analysing the data is the most logical continuing from the results gathered as the main aim was on conclusions of behavioural analysis. Discussions on the contribution to the wider body of knowledge will also form a part of this section. \par

\blank[line]

Primarily, the techniques for addressing the results is presented next. \par

\startsection[title=Techniques]
Two techniques will be used to address the results in this chapter. Firstly, the analytical approach with data analysis which lies on the hand, as it is the subject of the paper. \par
The second technique will be a reasoned evaluation of the findings itself. \par

Statistical analysis: Results, Analysis and Evaluation; mean time to hijack. \par
In relation to the database collected sensor geodata for particular routes with \infull{ADS-B}, see: \in{Chart}{.}[table:ads]. \par
\stopsection

% Total number of words per section: 73/350

\startsection[title=Comparison of the Results]
Comparing the results from the data in the experiments taken, it can be deduced that indeed the threshold of anomalies can be narrowed down to a 5\% margin for deviations from the baseline. \par
\stopsection

% Total number of words per section: 33/350

\startsection[title=Reflection]
In reflection of the project with the conducted case study of connected cars it has been a success in showing a novel approach on big data analysis for motion patterns in routes taken. \par
\stopsection

% Total number of words per section: 33/350

\startsection[title=Evaluation]
To evaluate the findings in the correlation of baseline behaviour presented in the design \in{Chapter}[chapter:design] the following method is used. \par
\stopsection

% Total number of words per section: 21/350

\startsection[title=Contribution to the Community]
The main contribution to the community is the novel aspect of dealing with cyber threats in real-time through estimation techniques that has not been used before for these scenarios. \par
\stopsection

% Total number of words per section: 29/350

\blank[line]

This chapter has summed up the results that were collected and analysed the data from the findings in order to show the contribution this paper has had for the domain of cyber security as a whole. \par
The project evaluation in form of a discussion is following next up. \par

\stopchapter

% Total number of words: 133/2000

\stopcomponent
