\startcomponent result
\product dissertation

\usemodule[filter]
\defineexternalfilter[R]
  [
    filtercommand={R CMD BATCH -q --save --restore \externalfilterinputfile\space \externalfilteroutputfile},
    output=\externalfilterbasefile.out,
    directory=stan,
    readcommand=\typefile,
    read=no,
    cache=yes,
  ]

\setuphead[chapter][textstyle={\tfd\sc}]
\startchapter[title={Results Collected and Analysis},reference=chapter:result]

% This section should assess your results/findings and your analyses of these.
% Examples of techniques you may use include reasoned evaluation, thorough product testing, scientific testing, data analysis, internal and external validation and statistical survey.
% You would normally make use of more than one technique, and your use of the techniques should reflect the postgraduate nature of the research.
% You should also justify your choice of techniques.
% This section also assesses your discussion on how your findings contribute to the wider academic body of knowledge, and any comparison of your results/findings/hypotheses with those of others.

Assessing the results collected and findings from the work done in the previous chapter, the reasoned evaluation and data analysis will be discussed here, which will be closed off with the validation of the data. \par
Analysing the data is the most logical continuing from the results gathered as the main aim was on conclusions of behavioural analysis. Discussions on the contribution to the wider body of knowledge will also form a part of this section. \par

\blank[line]

Primarily, the techniques for addressing the results are presented next. \par

\startsection[title=Techniques Used]
Two techniques will be used to address the results in this chapter. Firstly, the analytical approach that is presented with the data and the analysis of the data. The second technique will be a reasoned evaluation of the findings itself. \par
In relation to the data being used for analysis, the database with the collected geodata from the sensors of the system for the particular routes in the experiment via the \infull{ADS-B} can be seen in the \in{Chart}{.}[table:ads] (with the database dump at \in{Chart}{.}[table:gpx]). \par
Through the statistical data analysis, given it being the main subject of this paper, a comparison has been presented by using the same data-set to feed into the old machine learning technique described in \in{Section}[section:existingmodels] as well as the behavioural model this paper explained in detail (see \in{Section}[section:model]). Further comparisons have been shown between the different applications of the Bayesian estimation techniques (see \in{Appendix}[section:charts]). \par
The main outcome of these comparisons is to take the machine learning average of known-good routes as a baseline and compare the improvements of the behavioural analysis against that. Also the Bayesian techniques itself are compared with each other, taken the mean function with the estimation filters as a baseline to compare against and see whether the threshold of 5\% deviation can be achieved, thus improving the detection time of the anomalies as well as the precision in terms of the percentage representing the threshold being reached. \par
It has been shown, that the comparison between the signature based methods and the estimation filters indeed validate the assumption of major improvements in anomaly detection (see \in{Section}[section:resultscomparison]). \par
Reasoned evaluation on how the findings support the idea of a proactive approach with future state prediction is at hand. How the pattern based threat detection is more viable than signatures in themselves is also getting addressed here. \par
The improvements have been tied to metrics that can be compared with, in terms of time and precision, and therefore presenting a scientific evidence. \par
\stopsection

% Total number of words per section: 333/350

\startsection[title={Comparison of the Results},reference=section:resultscomparison]
Comparing the results with the data taken from the experiments, it can be deduced, that indeed the threshold of anomalies can be narrowed down to a 5\% margin for deviations from the baseline. \par
The comparisons in the plots, as shown in the examples of the Bayesian filters is as follows: \par
\startitemize[joinedup,nowhite]
\sym{»} Bayes filter plot with RStan: \\ \in{Chart}[chart:rstan] with a deviation prediction of {\bf 90\%} outside the {\bf 5\%} threshold
\sym{»} Bayesian sampling with PyMC3: \\ \in{Chart}[chart:pymc3] with a correlation of {\bf 87\%} inside the {\bf 5\%} threshold
\sym{»} Bayesian distribution and smoothing with PyBayes: \\ \in{Chart}[chart:pybayes] showing the mean function to compare against
\sym{»} Bayesian prediction with RStan: \\ \in{Chart}[chart:deviation] comparing three probabilistic functions to align the threshold
\stopitemize
The baseline for comparison is primarily the average function of genuine routes marked as known-good and the correlation to this line with the added correction of errors. It has been shown, that the behavioural prediction creates a function that is better aligned to this baseline, as well as, that it reduces the error correction to a smaller deviation threshold. \par
Further, the applications of the Bayes filter have also been compared with each other. Depending on the scenario, the sequential Bayesian filtering produced the best outcomes. All of the comparisons are attached to the Appendix section, at the end of this paper. \par
It has been consequently shown, that overall, with the same input of selected routes and the application of different methods for data analysis, the results on detecting anomaly behaviour differ in their outcome quite extensively. \par
\stopsection

% Total number of words per section: 247/350

\startsection[title=Reflection]
In reflection of the project with the conducted case study of connected cars, it has been a success in showing a novel approach on big data analysis for motion patterns of routes taken. \par
By looking at the current methods of machine learning (\in{Section}[section:existingmodels]), the application territory is for discovery type exploration of terrains with either the usage of artificial intelligence to deduce course of action or known-good/known-bad data feeding to navigate around the landscape. Although appropriate for navigating around pristine routes or for rescue missions, this falls short for addressing threats proactively. \par

The \goto{OpenCV}[url(http://opencv.org/)] (Open Source Computer Vision Library) open source library can be used for AI in terms of discovery, mapping and tracking: \par
\startcolumns[n=14,rule=off]
\externalfigure[https://marcosnietoblog.files.wordpress.com/2014/02/sample.png][height=30mm]
\column
\ 
\column
\ 
\column
\ 
\column
\ 
\column
\ 
\column
\ 
\column
\externalfigure[https://raw.githubusercontent.com/tomazas/opencv-lane-vehicle-track/master/img/screen2.png][height=30mm]
\column
\ 
\column
\ 
\column
\ 
\column
\externalfigure[http://www.jaychakravarty.com/wp-content/uploads/2012/04/lrfsim.png][height=30mm]
\column
\ 
\column
\ 
\stopcolumns

In sharp contrast, with the shown technique of predicting future states in motion patterns through correlation of historical data on the routes, instant analysis is possible to determine steering instructions for the selected routes. \par
The two specific scenarios used in this paper (\in{Section}[section:experiments]) validate the difference shown between discovery type machine learning and real-time correlation of anomalies through Bayesian estimation techniques. \par
In proving the assumption, that behavioural analysis will provide results to address cyber threats instantly and with high precision, specific data types were chosen as described in \in{Section}[section:datacollection]. Although the chosen data perfectly illustrated the results on the estimation techniques, the general idea for predictive modeling can be used on a variety of data-sets and shown with different types of transportation systems. \par
With choosing one specific example of cyber threat through remote exploitation described as hijacking, the representative risks have been selected. Further threat scenarios would underline the suitability of profile based analysis. \par
By looking ahead on benefiting off this research, these findings can be taken as an input for creating reactive based behaviour, and a whole new set of possibilities, coupled with requirements, will open up. \par
\stopsection

% Total number of words per section: 305/350

\startsection[title=Evaluation]
In order to evaluate the findings, in correlation to the baseline behaviour presented in the design (see \in{Chapter}[chapter:design]), the following method is used. \par
Metrics were chosen to look at the effectiveness on improving the detection algorithms via a time dimension and precision quotient. These correlations are expressed as: \par
$$ C_{i_1i_2\cdots i_n}(s_1,s_2,\cdots,s_n) = \langle X_{i_1}(s_1) X_{i_2}(s_2) \cdots X_{i_n}(s_n)\rangle $$ \par
It has to be stressed though, that these metrics do not depend on the data types chosen for the examples. On the other hand, these advancements in time to react do not provide enough time to act on collisions, which needs to be addressed in a different way, as the physical implications of getting to a halt is more complex than computer instructions. \par
The validation of the results was achieved through the demonstration in such, that improvement on the metrics was accomplished through the increase of the time-frame indicator and the precision in which the deviation of malicious events correlate. \par
Looking specifically at the mean time to hijack from the cyberspace and the possible time gained to trigger actions for cyber resilience, it can be concluded, that cyber defence was achieved with the algorithms built in this paper. \par
In summary, it has been shown, that effective behaviour prediction enhances threat leads and detection of malicious intruders.
Comparing the results and findings with the current available models (\in{Section}[section:resultscomparison]), it can be concluded, that the hypothesis, which does not differ from the contemporary methods, was proven. \par
To close, the conclusive elaboration on the results, analysis and evaluation has been presented and justified with the case study on connected cars (\in{Section}[section:casestudy]). \par
\stopsection

% Total number of words per section: 271/350

\startsection[title=Contribution to the Community]
The paper's contribution to the community is measured against the current, peer-reviewed articles on the topic of cyber threats targeted at connected cars. \par
The following 3 published journals, in the order of importance, have been chosen in regards to the research this paper has conducted: \par
\startitemize[joinedup,nowhite]
\sym{»} {\it "Potential Cyberattacks on Automated Vehicles"} -- \cite[authoryear][6899663]
\sym{»} {\it "Defending Connected Vehicles Against Malware: Challenges and a Solution Framework"} -- \cite[authoryear][6720160]
\sym{»} {\it "Connected cars – the next targe tfor hackers"} -- \cite[authoryear][Ring201511]
\stopitemize
The main paper being the first one, describing the attack vectors in automated vehicles and how to defend against those. The other two publications are covering the current signature based techniques used in connected cars in order to present a solution to the cyber problem. They also highlight the fact, that security is bolted onto the design of these systems and thus making it an attractive target. \par
To put this paper into perspective in terms of the contribution to the community, it compared the illustrated novel idea of using estimation techniques for anomaly detection against the methods described in the publications above. The outcome of this comparison is an increased window to react to cyber threats and precision expressed in a 5\% deviation from the norm which has been demonstrated. \par
The framework around the solution together with the threat landscape will stay the same and only the data analysis will be exchanged for the methods of predictive modeling. \par
Dealing with cyber threats through this novel aspect of future state behavioural analysis, the gains will materialise in the ability to detect anomalies in real-time, which could not be achieved previously for the chosen scenarios. \par
Comparing and contrasting the presented estimation techniques to the machine learning approach, summarised in the selected articles, the main contribution to the community will be further submitted to a journal for a peer-review. \par
\stopsection

% Total number of words per section: 270/350

\blank[line]

This chapter has summed up and validated the results that were collected and analysed the data from the findings in order to show the contribution this paper has had for the domain of cyber security as a whole. \par
The project evaluation in form of a discussion is following up next. \par

\stopchapter

% Total number of words: 1561/2000

\stopcomponent
