\startcomponent result
\product dissertation

\usemodule[filter]
\defineexternalfilter[R]
  [
    filtercommand={R CMD BATCH -q --save --restore \externalfilterinputfile\space \externalfilteroutputfile},
    output=\externalfilterbasefile.out,
    directory=stan,
    readcommand=\typefile,
    read=no,
    cache=yes,
  ]

\setuphead[chapter][textstyle={\tfd\sc}]
\startchapter[title={Results Collected and Analysis},reference=chapter:result]

% This section should assess your results/findings and your analyses of these.
% Examples of techniques you may use include reasoned evaluation, thorough product testing, scientific testing, data analysis, internal and external validation and statistical survey.
% You would normally make use of more than one technique, and your use of the techniques should reflect the postgraduate nature of the research.
% You should also justify your choice of techniques.
% This section also assesses your discussion on how your findings contribute to the wider academic body of knowledge, and any comparison of your results/findings/hypotheses with those of others.

Assessing the results collected and findings from the work done in the previous chapter, the reasoned evaluation and data analysis will be discussed here, that will be closed off with the validation of the data. \par
Analysing the data is the most logical continuing from the results gathered as the main aim was on conclusions of behavioural analysis. Discussions on the contribution to the wider body of knowledge will also form a part of this section. \par

\blank[line]

Primarily, the techniques for addressing the results is presented next. \par

\startsection[title=Techniques]
Two techniques will be used to address the results in this chapter. Firstly, the analytical approach that is presented with the data and the analysis of the data. The second technique will be a reasoned evaluation of the findings itself. \par
In relation to the data being used for analysis, the database with the collected geodata from the sensors of the system for the particular routes in the experiment via the \infull{ADS-B} can be seen in the \in{Chart}{.}[table:ads]. \par
With the statistical data analysis, also the main subject of this paper, a comparison has been presented by using the same data-set to feed into the old machine learning technique described in \in{Section}[section:existingmodels] as well as the behavioural model this paper explained in detail (see \in{Section}[section:model]). Further comparisons have been shown between the different applications of the Bayesian estimation techniques (see \in{Appendix}[section:charts]). \par
The main outcome of these comparisons is to take the machine learning average of known-good routes as a baseline and compare the improvements of the behavioural analysis against that. Also the Bayesian techniques itself are compared with each other, taken the mean function with the estimation filters as a baseline to compare against and see whether the threshold of 5\% deviation can be achieved, thus improving the detection time of the anomalies as well as the precision in terms of the percentage representing the threshold being reached. \par
It has been shown that the comparison between the signature based methods and the estimation filters indeed validating the assumption of major improvements in anomaly detection (see \in{Section}[section:resultscomparison]). \par
Reasoned evaluation on how the findings support the idea of a proactive approach with future state prediction is at hand. How the pattern based threat detection is more viable than signatures in itself is also getting addressed here. \par
The improvements have been tied to metrics that can be compared with in terms of time and precision and therefore presenting a scientific evidence. \par
\stopsection

% Total number of words per section: 324/350

\startsection[title={Comparison of the Results},reference=section:resultscomparison]
Comparing the results from the data in the experiments taken, it can be deduced that indeed the threshold of anomalies can be narrowed down to a 5\% margin for deviations from the baseline. \par
\stopsection

% Total number of words per section: 33/350

\startsection[title=Reflection]
In reflection of the project with the conducted case study of connected cars it has been a success in showing a novel approach on big data analysis for motion patterns in routes taken. \par
\stopsection

% Total number of words per section: 36/350

\startsection[title=Evaluation]
To evaluate the findings in the correlation of baseline behaviour presented in the design \in{Chapter}[chapter:design] the following method is used. \par
These scenarios highlight the difference between discovery machine learning and real-time correlation of anomalies. \par
Results, Analysis and Evaluation; mean time to hijack. \par
Validating the results. \par
\stopsection

% Total number of words per section: 21/350

\startsection[title=Contribution to the Community]
The paper's contribution to the community is measured against the current peer-reviewed articles on the topic of cyber threats against connected cars. \par
The following 3 published journals, in the order of importance, have been chosen in regards to the research this paper has conducted: \par
\startitemize[joinedup,nowhite]
\sym{»} {\it "Potential Cyberattacks on Automated Vehicles"} -- \cite[authoryear][6899663]
\sym{»} {\it "Defending Connected Vehicles Against Malware: Challenges and a Solution Framework"} -- \cite[authoryear][6720160]
\sym{»} {\it "Connected cars – the next targe tfor hackers"} -- \cite[authoryear][Ring201511]
\stopitemize
The main paper being the first one, describing the attack vectors in automated vehicles and how to defend against those. The other two publications are covering the current signature based techniques used in connected cars in order to present a solution to the cyber problem. They also highlight the fact that security is bolted onto the design of these systems and thus making it an attractive target. \par
To put this paper into perspective in terms of the contribution to the community, it will compare the presented novel idea of using estimation techniques for anomaly detection against the methods described in the publications above. \par
The framework around the solution together with the threat landscape will stay the same and only the data analysis will be exchanged for the methods of predictive modeling. \par
Dealing with cyber threats through this novel aspect of future state behavioural analysis the gains will materialise in the ability to detect anomalies in real-time which could not be achieved before for the chosen scenarios. \par
Comparing and contrasting the presented estimation techniques to the machine learning approach, summarised in the selected articles, the main contribution to the community will be further submitted for a peer-review. \par
\stopsection

% Total number of words per section: 240/350

\blank[line]

This chapter has summed up the results that were collected and analysed the data from the findings in order to show the contribution this paper has had for the domain of cyber security as a whole. \par
The project evaluation in form of a discussion is following up next. \par

\stopchapter

% Total number of words: 133/2000

\stopcomponent
