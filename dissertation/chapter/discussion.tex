\startcomponent discussion
\product dissertation

\setuphead[chapter][textstyle={\tfd\sc}]
\startchapter[title=Discussion,reference=chapter:discussion]
Looking back to the aims of this paper, this chapter is going to discuss the achievements laid out in the objectives for the project. By analysing the quality of the whole project and assessing how appropriate the critical literature review supported these aims it will be determined whether relevant techniques were missed out upon. \par
Further elaboration will highlight whether the hypothesis was chosen correctly and whether a different evaluation criteria could have been more supportive.
Guiding through the case study the extent of completeness in regards of the experimental results will be shown and finished off with the reliability of the conclusions. \par
Further two topics will be touched in this chapter about the effectiveness of the correlation and the ethical considerations for the change in behaviour with connected transportation systems introducing reactive decision making. \par

\startsection[title=Project Evaluation]
% This should evaluate the quality of the project as a whole, including a consideration of how well each of the objectives were met.
% How effective was the literature review? Is it possible that there were relevant techniques or issues that were ignored?
% Were there alternative hypotheses that could have been tested?
% For developmental projects, what other evaluation criteria could have been considered?
% How complete were the experimental results; how reliable are the conclusions

The evaluation of the project is measured in the quality the paper as a whole was able to deal with in order to show a novel approach on proactive cyber threat detection. \par
The following list shows the objectives that have been set out in \in{Section}[section:objectives] and whether they have been met: \par

{\tfx
\startitemize[joinedup,nowhite]
\sym{\definefont[djv][file:DejaVuSans.ttf] {\startcolor[darkgreen]\djv☑\stopcolor}} Criteria for defining autonomous connected cars (sub-set of \infull{CPS})
\sym{\definefont[djv][file:DejaVuSans.ttf] {\startcolor[darkgreen]\djv☑\stopcolor}} Creation of scenarios (direction, speed, location, time) and data format proposition on behavioural profiling
\sym{\definefont[djv][file:DejaVuSans.ttf] {\startcolor[darkgreen]\djv☑\stopcolor}} Comparison of different density functions, particularly Bayesian estimation algorithms for anomaly detection
\sym{\definefont[djv][file:DejaVuSans.ttf] {\startcolor[darkgreen]\djv☑\stopcolor}} Creation of an algorithm for predicting future states with estimation techniques
\sym{\definefont[djv][file:DejaVuSans.ttf] {\startcolor[darkgreen]\djv☑\stopcolor}} Validation of predicted values on historical data in predefined simulations
\sym{\definefont[djv][file:DejaVuSans.ttf] {\startcolor[darkgreen]\djv☑\stopcolor}} Suggestions of fail-safe actions in autonomous systems in the event of a breach
\sym{\definefont[djv][file:DejaVuSans.ttf] {\startcolor[darkgreen]\djv☑\stopcolor}} Presenting improvements through behavioural analysis in comparison to signature based machine learning
\sym{\definefont[djv][file:DejaVuSans.ttf] {\startcolor[darkgreen]\djv☑\stopcolor}} Proposition of a live warning and tracking system to form a control centre against cyber attacks
% \sym{\definefont[djv][file:DejaVuSans.ttf] {\startcolor[orange]\djv☐\stopcolor}}
% \sym{\definefont[djv][file:DejaVuSans.ttf] {\startcolor[darkred]\djv☒\stopcolor}}
\stopitemize}

The first two objectives are more of an input into the project and necessary in order to process and show analytical results. Hence the level of achievement was more on the technicality rather than the analytical work. Given the focus of this paper was on providing evidence on the superiority of behavioural analysis, the suggestions for further study and fail-safe actions were only briefly touched and could be elaborated in more depth. \par
Taking the reader back to the literature review, which was structured around the founding blocks of the theory of behavioural analysis it demonstrated the relevancy of the research question and covered all contemporary techniques for estimation prediction. Whether there are even more appropriate algorithms that could be developed needs to be seen, but given the recent research on the subject any viable option was incorporated. Therefore, it can be concluded that the critical review was appropriate and effective. \par
The hypothesis was chosen rightly by testing and proving that predictive models can detect cyber attacks proactively and an alternative hypothesis would not be appropriate. \par
Reminding ourselves with the evaluation criteria that was to compare a new behavioural analysis technique with current signature based machine learning the aspects on time and precision was chosen and the only other aspect could have been the preservation of life. \par
With the experiments being complex and sophisticated, only two simulations were conducted and it would be more desirable to validate on a more broader scale, but having said that, the results were conclusive and reliable in achieving to prove the theory of applying density functions to correlate and with that detecting anomalies. \par
\stopsection

% Total number of words per section: 320/300

\startsection[title=Correlation Effectiveness]
It has been shown that the target on the \infull{IoV} is increasingly lucrative provided with the distribution of systems and malicious intent or terror related threats will turn the focus on exploiting their weaknesses. \par
Behavioural changes in autonomous driving creates a new set of valid routes and patterns of motions that could differentiate between manual and autonomous driving. \par
Looking at the correlation and more specifically the effectiveness of the correlation to detect anomalies with the profile deviation of mean group behaviour has shown that the error rate can be reduced as much as to 5\%. Moreover, the rates in false-positive alarms is significantly reduced providing a more accurate and precise prediction. \par
The creation of a correlation between an intended cyber threat and the threshold of deviation has been discussed and presented to prove effectiveness. \par
\stopsection

% Total number of words per section: 136/300

\startsection[title=Ethical Considerations]
Even though the project itself had no ethical considerations in regards to the experiments conducted, there are however ethical aspects with the outcome of automated systems that needs to be discussed or the least to be pointed out. \par
The first practical aspect is about the responsibility and accountability of fully autonomous systems and the implications their actions create. Will insurance cover the decisions taken by connected vehicles? Who will be the owner of the vehicles in a society associated with a sharing economy. \par
The more pressing question is how to program (instruct) the vehicle on reacting to events. What are the moral, ethical and philosophical duties in case of unavoidable collisions \footnote{\goto{justiceharvard.org}[url(http://justiceharvard.org/)]}? Do collision control algorithms need to weight on peoples life \cite[authoryear][sandel2011justice]? \par
Ultimately, there will be a change of behaviour that can be observed with the increase of fully autonomous vehicles as they have a predefined way of driving. How will that affect the passenger who is not in charge of the control of the vehicle any more? Lastly, there has to be evaluated whether and how much of a reduction in so-called "pilot error" incidents the society will accept statistically. \par
\stopsection

% Total number of words per section: 190/300

\blank[line]

By discussing the project evaluation above with pointing to correlation effectiveness as well as ethical considerations the following chapter will give closure to this paper with the conclusions of the project and directing to further research opportunities. \par

\stopchapter

% Total number of words: 817/1000

\stopcomponent
