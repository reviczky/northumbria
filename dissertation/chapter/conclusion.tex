\startcomponent conclusion
\product dissertation

\startchapter[title={Conclusion \& Suggestions for Further Study},reference=chapter:conclusion]
% The conclusions should present the answer to the original research question, along with any other conclusions reached along the way (for example, about the best choice of tools or technologies).
% There should be no 'surprises' in this chapter: each conclusion should have been noted and evidenced elsewhere in the dissertation.
% Recommendations should also be included for further research, for any possible practical applications, or any recommendations for future practice.

% This section should assess the strength and presentation of the findings and the recommendations.
% You should present your answer to the original hypothesis/research question and should discuss how well your original research aim has been met and the original problem has been solved.
% This should include reference to the results obtained by you.
% You should also discuss any interesting additional discoveries that have been made.
% Recommendations may include suggestions for further research, suggestions for improved practice based on your findings and suggestions for practical application of any new concepts that have been investigated.

Concluding on the work that was done through the proposed research question set out at the start of the project (\in{Section}[section:researchquestion]) all the way along the construction of this paper, each of the following sections will arrive to the conclusions in reference to the project. \par
Starting off by recapping the problem statement (\in{Section}[section:problemstatement]) and reviewing the outlined objectives of the project (\in{Section}[section:objectives]) the content of this section will move on to summarise the reflection of the conducted work before mapping out the results (\in{Chapter}[chapter:result]) that have been found. \par
The chapter will finally look at the further research (\in{Section}[section:futurework]) options and also propose practical implementations on threat prediction as well as suggesting ways of application for warning and tracking systems. \par

\blank[line]

Thus, the conclusions for this paper using Bayesian estimation and tracking for threat prevention in connected cars \cite[authoryear][haug2012bayesian] are as follows. \par

\startsection[title=Addressing the Problem Statement]
Before setting out and addressing the problem of cyber threat detection the landscape of current solutions within the field of connected devices had been looked at. Whilst semi-autonomous vehicles do address partially the challenges of cyber-attacks it was presented that the way they are implementing defences are exclusively signature based machine learning algorithms. Clearly there is a lack of cyber resilience in current connected vehicles. \par
Therefore, in order to address the problem of proactive anomaly detection it has been shown that by replacing machine learning with behavioural analysis can achieve built-in security for \infull{CPS} and for that the use case of fully autonomous connected cars have been taken as a subject of a series of experiments. \par
With the specific scenario of cyber hijacking and change of routes the threat modelling of remote hacking and cyber espionage was highlighted as potential dangerous intrusion events and it has been presented that behaviour analysis and profiling can be the answer to solve this deficiency. \par
To achieve a complete proactive approach on built-in security suggestions have been made on the on-the-fly reaction models that tackle the incidents uncovered with anomaly detection. \par
Proving that real-time threat detection is a problem that can be successfully addressed this paper demonstrated the theory on a sub-set of \infull{CPS} by observing connected cars and their autonomous behaviour with the help of sensor data that included geolocation, speed and cardinal direction. \par
In conclusion, the current methods of machine learning found in contemporary vehicles are unpractical when it comes to cyber-attacks and detecting anomalies. Substituting that with profiling based behavioural analysis tackles the problem statement in a way it should have been addressed in the first place. \par
\stopsection

% Total number of words per section: 279/350

\startsection[title=Review of the Objectives]
Looking back at the project's objectives, it can be concluded that the framework on defining connected cars as the specific sub-set of \infull{CPS} to be used as a use case was rightly chosen and explained. The scenarios that have been carefully chosen to demonstrate both the deficiency in current threat prediction models as well as proving the hypothesis around behavioural analysis through densitiy functions were conclusive. \par
Great efforts have been put into coming up with a global and open data-set for capturing motion patterns of connected cars and this should be taken further to establish an industry-wide standard for big data analysis. \par
On the comparing and contrasting methods and advancements in the automotive industry the aims of demonstrating the superiority and more importantly the suitability of Bayesian estimation techniques on anomaly detection were also accomplished. \par
By being able to predict future states of the vehicles a whole new set of opportunities opened up and could be utilised apart from cyber threat detection. The recommendations are described in the section of future work below (see \in{Section}[section:futurework]). \par
The validation of the data shown in the simulations underline the scientific proof of the improvements achieved with the novel way of analysing live data from sensor readings. \par
Conclusions on the proposed further research to create a foundation on future work that was part of the objectives is explained and elaborated on in the section designed to look at the application of the findings (see \in{Section}[section:futurework]). \par
Overall, it can be deduced that all of the objectives have been met and a holistic approach was given on tackling cyber-attacks and anomaly detection within connected cars in order to have security built in. \par
With this, in proving the original research question that had been broken down into the mentioned objectives it has been demonstrated how the problem statement was successfully addressed. \par
\stopsection

% Total number of words per section: 305/350

\startsection[title=Reflection on the Work Done]
% "Bayesian Estimation and Tracking: A Practical Guide" \cite[authoryear][haug2012bayesian]
% A practical approach to estimating and tracking dynamic systems in real-worl applications
% describes effective numerical methods for evaluating density-weighted integrals, including linear and nonlinear Kalman filters for Gaussian-weighted integrals and particle filters for non-Gaussian cases

The vast majority of the work has concentrated on the specific issue of estimation techniques and their advantages over signature based algorithms. One of the main literature on practical application on Bayesian estimation \cite[authoryear][haug2012bayesian] shaped the numerical approach that was taken on evaluating the various density functions, including the Kalman and Gaussian filters. \par
With the findings from the literature review the choice of the appropriate set of tools and technologies that were implemented on these theories were utilised based on their peer acceptance and openness. Supported by that, the python and R based solutions were then successfully applied and demonstrated on the chosen use case of connected cars with the pre-setttings of experimental scenarios. \par
In the next steps the work that had been done was to map the project objectives to technical deliverables and observing the results, keeping an eye on the expected outcomes. \par
The creation of the behavioural algorithm that has followed (\in{Section}[section:model]) shows the complexity that was needed to achieve in solving the problem statement. Reflecting on the main parts of the hard work in using a data input from connected cars, structuring those inputs and conducting big data analytics to produce results that could then be analysed and verified, comprises the bulk of technical work that has been achieved. \par
The details in the design phase which enabled to run the experiments of the case study that led to an algorithm to determine deviations from the norm with a margin of 5\% has been shown in \in{Chapter}[chapter:result]. \par
Also a vital part of the work was the engagement with the wider community to see the requirements on autonomous systems and what challenges had been uncovered previously. \par
Having had the work completed, the focus of attention was on the collection and analysis of the results, to demonstrate and validate the cyber resilience by correlating various functions and conclude with an acceptable determination of whether or not that function represents an anomaly. \par
\stopsection

% Total number of words per section: 321/350

\startsection[title=Mapping of the Results]
Turning onto the results collected earlier in the paper (\in{Chapter}[chapter:result]) with the effectiveness by which these results were mapped back to the objectives and eventually to the problem statement the next few sentences will be summing up the achieved goals. \par
Reflecting on the analysis and evaluation of the correlation functions on behaviour profiling that lead to a conclusion in which indeed it is possible to almost immediately decide whether a predicted future state is within thresholds of normal behaviour. This gives a huge advantage in timing compared to the signature based machine learning on the example of mean time to hijack, which usually needs a significant amount to baseline (usually around 3 months) and depends purely on feeding in valid and genuine routes. As discussed earlier, these legacy techniques are not adequate for threat prevention but rather reserved for exploration of new environments. \par
Moreover, the results show that the technique of profiling reveals a lot about the systems and their behaviour as well. Abstracting the theory which was shown on the example of connected cars and with specific sensor reading data, the model itself can be used more widely and on a variety of scenarios for detecting anomalies, not just cyber threats. \par
Revisiting the results on the case study for conclusion it can be noted that even though they are very much representative and verified, further case studies can reveal segmented deviations for specific systems that could be explored in more depth. To support that, it should be noted that the table of ADS-B (see \in{Table}[table:ads]) is very helpful for analysing correlations of various \infull{CPS}. \par
In the next section these contributions will be emphasised to encourage the community to build on these findings and methods. Forward looking to the applicability of the results for reactive based models, the advantage in time and precision that was achieved will surely be very valuable for the automotive industry in particular to explore further. \par
\stopsection

% Total number of words per section: 323/350

\startsection[title={Further Work},reference=section:futurework]
% Recommendations should also be included for further research, for any possible practical applications, or any recommendations for future practice.

Now that the problem statement has been addressed, the objectives that have been set out met and through the elaboration of the work that this paper documented the conclusion be drawn that the results prove the hypothesis, the attention is shifting now towards the opportunities these results will unleash and enable to be used for other projects. \par
Since the research question on using Bayesian estimation for threat detection has been demonstrated, further work can be done on looking at optimising data collection with sensor values that is available to detect anomalies. Physical system specific solutions could enhance resilience for targeted transportation vehicles. \par
Recommendations on utilising the findings in this project is not limited but shown and included in the list below: \par
\startitemize[joinedup,nowhite]
\sym{»} The development of decision and reaction making algorithms, including \infull{AI} (AI) in order to instruct the system on how to counter-attack the cyber threat
\sym{»} Defining fail-save stopping, safe disconnection and shutdown when breached; this could also include scenarios of attacking the national grid or the internet
\sym{»} A practical application of a global tracking and warning system (similar to the ones used in the aviation industry) would enhance the security and traffic management
\sym{»} Incorporating these techniques into a centralised control centre for cyber-attacks (including ITS) would enable the analysis to be conducted on a holistic level
\stopitemize
With the methods of detecting anomalies in data flows in fully autonomous connected cars, future practice could be applicable for land, sea and air-based transportation systems to be automated by addressing cyber security aspects. \par
Another angle of research could be also to look at human behavioural patterns versus fully level 5 autonomous system behaviour, as human behaviour tends to be more creative and unpredictable in contrast to programmed system procedures. \par
On legal and legislative topics there are many open questions around personal data, data privacy and data protection as well as ethical implications on the automated decision making which could be addressed. \par
\stopsection

% Total number of words per section: 319/350

\blank[line]

This final chapter has looked at the project's conclusion by answering the research question posed at the beginning of the paper on proactive cyber threat detection of \infull{CPS} using Bayesian estimation.
Through the case study on connected cars it has been shown that a novel way of dealing with anomalies can be introduced and that cyber defence is achievable. \par
Suggestions have been made for continuing the work based on the findings presented in this paper and a clear path of taking this research further has been shown. \par
With this, the formal part of the thesis is closed. \hfill \blacksquare \par

\stopchapter

% Total number of words: 1779/2000

\stopcomponent
