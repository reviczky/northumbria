\startcomponent conclusion
\product dissertation

\startchapter[title={Conclusion \& Suggestions for Further Study},reference=chapter:conclusion]
% The conclusions should present the answer to the original research question, along with any other conclusions reached along the way (for example, about the best choice of tools or technologies).
% There should be no 'surprises' in this chapter: each conclusion should have been noted and evidenced elsewhere in the dissertation.
% Recommendations should also be included for further research, for any possible practical applications, or any recommendations for future practice.

Concluding on the work that was done through the proposed research question set out at the start of the project all the way along the build-up of this paper, each of the following sections will arrive to the conclusions in reference to the project. \par
Starting off by recapping the problem statement and reviewing the presented/set-out objectives of the project and then followed by the reflection of the conducted work before mapping out the results found. \par
The chapter will finally look at the further research option and proposed practical implementations on threat prediction and suggesting/recommending applications on warning and tracking. \par

\blank[line]

The conclusions are \par
Conclusions and Recommendations \cite[authoryear][haug2012bayesian] \par
http://eu.wiley.com/WileyCDA/WileyTitle/productCd-0470621702.html

\startsection[title=Addressing the Problem Statement]
Addressing the problem if proactive cyber threat detection. \par
\stopsection

% Total number of words per section: 350

\startsection[title=Review of the Objectives]
Review of the objectives \par
Answer to the original research question \par
mapping of the objectives \par
\stopsection

% Total number of words per section: 350

\startsection[title=Reflection on the Work Done]
literature review, tools and technologies \par
Reflection on the project \par
Choice of tools and technologies \par
\stopsection

% Total number of words per section: 350

\startsection[title=Mapping of the Results]
Objectives?/case study/contribution? \par
Summary \par
\stopsection

% Total number of words per section: 350

\startsection[title=Further Work]
% Final section?
% Recommendations should also be included for further research, for any possible practical applications, or any recommendations for future practice.

With the solution of detecting anomalies \par
Recommedation \par
Practical Application \par
Further work \par
Future practice \par
Warning System and Tracker (flightradar) \par
\stopsection

% Total number of words per section: 350

\blank[line]

This final chapter has looked at the project's conclusion by answering the research question posed at the beginning of the paper on proactive cyber threat detection of \infull{CPS} using Bayesian estimation.
Through the case study on connected cars it has been shown that a novel way of dealing with anomalies can be introduced and that cyber defence is achievable. \par
Suggestions have been made for continuing the work based on the findings presented in this paper and a clear path of taking this research further has been shown. \par
With this, the formal part of the thesis is closed. \hfill \blacksquare \par

\stopchapter

% Total number of words: 98/2000

\stopcomponent
