\startcomponent conclusion
\product dissertation

\startchapter[title={Conclusion \& Suggestions for Further Study},reference=chapter:conclusion]
% The conclusions should present the answer to the original research question, along with any other conclusions reached along the way (for example, about the best choice of tools or technologies).
% There should be no 'surprises' in this chapter: each conclusion should have been noted and evidenced elsewhere in the dissertation.
% Recommendations should also be included for further research, for any possible practical applications, or any recommendations for future practice.

Concluding on the work that was done through the proposed research question set out at the start of the project all the way along the construction of this paper, each of the following sections will arrive to the conclusions in reference to the project. \par
Starting off by recapping the problem statement and reviewing the outlined objectives of the project the content of this section will move on to summarise the reflection of the conducted work before mapping out the results that have been found. \par
The chapter will finally look at the further research options and also propose practical implementations on threat prediction as well as suggesting ways of application for warning and tracking systems. \par

\blank[line]

Thus, the conclusions for this paper using Bayesian estimation and tracking for threat prevention in connected cars \cite[authoryear][haug2012bayesian] are as follows. \par

\startsection[title=Addressing the Problem Statement]
Addressing the problem if proactive cyber threat detection. \par
\stopsection

% Total number of words per section: 350

\startsection[title=Review of the Objectives]
Review of the objectives \par
Answer to the original research question \par
mapping of the objectives \par
\stopsection

% Total number of words per section: 350

\startsection[title=Reflection on the Work Done]
% "Bayesian Estimation and Tracking: A Practical Guide" \cite[authoryear][haug2012bayesian]
% A practical approach to estimating and tracking dynamic systems in real-worl applications
% describes effective numerical methods for evaluating density-weighted integrals, including linear and nonlinear Kalman filters for Gaussian-weighted integrals and particle filters for non-Gaussian cases
literature review, tools and technologies \par
Reflection on the project \par
Choice of tools and technologies \par
\stopsection

% Total number of words per section: 350

\startsection[title=Mapping of the Results]
Objectives?/case study/contribution? \par
Summary \par
\stopsection

% Total number of words per section: 350

\startsection[title=Further Work]
% Recommendations should also be included for further research, for any possible practical applications, or any recommendations for future practice.

Now that the problem statement has been addressed, the objectives that have been set out met and through the elaboration of the work that this paper documented the conclusion be drawn that the results prove the hypothesis, the attention is shifting now towards the opportunities these results will unleash and enable to be used for other projects. \par
Since the research question on using Bayesian estimation for threat detection has been demonstrated, further work can be done on looking at optimising data collection with sensor values that is available to detect anomalies. Physical system specific solutions could enhance resilience for targeted transportation vehicles. \par
Recommendations on utilising the findings in this project is not limited but shown and included in the list below: \par
\startitemize[joinedup,nowhite]
\sym{»} The development of decision and reaction making algorithms, including \infull{AI} (AI) in order to instruct the system on how to counter-attack the cyber threat
\sym{»} Defining fail-save stopping, safe disconnection and shutdown when breached; this could also include scenarios of attacking the national grid or the internet
\sym{»} A practical application of a global tracking and warning system (similar to the ones used in the aviation industry) would enhance the security and traffic management
\sym{»} Incorporating these techniques into a centralised control centre for cyber-attacks (including ITS) would enable the analysis to be conducted on a holistic level
\stopitemize
With the methods of detecting anomalies in data flows in fully autonomous connected cars, future practice could be applicable for land, sea and air-based transportation systems to be automated by addressing cyber security aspects. \par
Another angle of research could be also to look at human behavioural patterns versus fully level 5 autonomous system behaviour, as human behaviour tends to be more creative and unpredictable in contrast to programmed system procedures. \par
On legal and legislative topics there are many open questions around personal data, data privacy and data protection as well as ethical implications on the automated decision making which could be addressed. \par
\stopsection

% Total number of words per section: 319/350

\blank[line]

This final chapter has looked at the project's conclusion by answering the research question posed at the beginning of the paper on proactive cyber threat detection of \infull{CPS} using Bayesian estimation.
Through the case study on connected cars it has been shown that a novel way of dealing with anomalies can be introduced and that cyber defence is achievable. \par
Suggestions have been made for continuing the work based on the findings presented in this paper and a clear path of taking this research further has been shown. \par
With this, the formal part of the thesis is closed. \hfill \blacksquare \par

\stopchapter

% Total number of words: 232/2000

\stopcomponent
