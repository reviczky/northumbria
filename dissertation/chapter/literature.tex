\startcomponent literature
\product dissertation

\startchapter[title={Critical Review of the Literature},reference=chapter:literature]
% https://patthomson.net/2014/01/16/connecting-chapterschapter-introductions/

The literature review in this chapter is looking at the relevant contemporary papers, books and articles related to the project and gives definitions as well as a frame to the subject of behavioural analysis and why the research question (\in{Section}[section:researchquestion]) is justified. \par
Building a logical flow, the sections will cover the validity of the case study of connected cars followed by the review of existing models on data analysis and solutions utilised in the field. Continuing with the elaboration on the data capturing techniques for \infull{CPS} and different takes on system profiling models that will make up behavioural data with references to state-of-the-art research will lead to answering questions on how to deal with anomaly detection. In order to relate to current cyber threats and giving evidence of the contribution this project provides, the closing critique will be on the topic of threat prediction methods. \par

\blank[line]

% Relevancy to the paper, work done, results found, strenghts of findings and critical evaluation
Each section explains the relevancy to this paper, discussing the work that has been done containing the results found and the strengths of these findings. All of the literature presented will be critically evaluated and linked back to the project. \par

\startsection[title=Usability Study]
% Relevancy
As the paper is proving the research question on a case study of connected cars, this section will dive into the importance of the subset of \infull{CPS} and what autonomous and semi-autonomous control stands for, how far remote control can be taken. Through the comparison of literature on estimation techniques and signature based learning the gathered intelligence is linked to the new threat landscape of cyber-attacks. \par
% Work done & Results found
Even though this research is touching the whole space of \infull{CPS}, systems that {\it\quote{interface directly with the physical world, making the detection of environmental changes and the system behavior adaptation to be considered the key challenge}} \cite[authoryear][krishna2014challenges], the hypothesis of proactive approach is proven on a specific sub-set of vehicles, namely connected cars. \par
As connected cars will be the building blocks of the emerging \infull{IoV} (IoV) \cite[authoryear][6823640] it is a perfect example to conduct the experiments on for behaviour profiling. \par
The different levels of automation between manually driven, semi-autonomous and fully autonomous connected vehicles, that can navigate themselves without any human interaction or input is described and explained in the book of \quote{Preparing a Nation for Autonomous Vehicles} \cite[authoryear][Fagnant2015167]. For this case study the baseline will be set on level 5, fully autonomous connected cars. \par
Communication channels and flow of communication between \infull{CPS} is shown in \in{Figure}[figure:cps]. Looking at the importance of profiling autonomous vehicles with deduced behaviour like vehicle usage, see \in{Figure}[figure:vehicleuse], the comparable profiles build the base of correlating behavioural analysis, so that norming against general behaviour will give the mean against which the deviation will be calculated. \par
In order to put the entire simulation into perspective of tackling a proactive approach on detecting anomalies, these behaviours will be analysed with the different sequential Bayesian estimation techniques as seen in \in{Section}[section:anomalydetection]. \par
Having had the specific set of vehicles defined for this project, the type of profiling and behaviour on navigation explained, the very last aspect on why all this is relevant is the real problem of cyber threats that these autonomous vehicles will be exposed and are facing from the internet \cite[authoryear][6803166]. This brings a whole new set of challenging situations \cite[authoryear][adouane2016autonomous], like the particular one on cyber-hijacking that has to be dealt with. \par
%\quote{Unlike the traditional embedded systems, the CPSs interface directly with the physical world, making the detection of environmental changes and the system behavior adaptation to be considered the key challenges in the design of such systems.} \cite[authoryear][krishna2014challenges]
%\quote{Moreover, connected vehicles are also the building blocks of emerging Internet of Vehicles (IoV).} \cite[authoryear][6823640]
%\quote{Some companies have pushed the envelope even further by creating almost fully autonomous vehicles (AVs) that can navigate highways and urban environments with almost no direct human input.} \cite[authoryear][Fagnant2015167]
%\quote{This book reveals innovative control architectures that can lead to fully autonomous vehicle navigation in these challenging situations.} \cite[authoryear][adouane2016autonomous]
%\quote{The next step in this evolution is just around the corner: the Internet of Autonomous Vehicles.} \cite[authoryear][6803166]

\blank[line]

\startcolumns[n=2,rule=on]
\startplacefigure[title={CPS Communication \\ \tfx © Cyber Physical Systems (CPSs)},list={Cyber-Physical System Communication},reference=figure:cps]
 \externalfigure[http://www.igi-global.com/sourcecontent/9781466673120_112206/978-1-4666-7312-0.ch001.f01.png][height=35mm]
\stopplacefigure
\column
\startplacefigure[title={Vehicle Use \\ \tfx © Preparing a Nation for Autonomous Vehicles},list={Vehicle Use for Autonomous Vehicles},reference=figure:vehicleuse]
 \externalfigure[https://ai2-s2-public.s3.amazonaws.com/figures/2016-11-08/192c9fe684f2340d413e7826ac574d16595c2fdb/10-Figure1-1.png][method=png,height=35mm]
\stopplacefigure
\stopcolumns

By choosing \infull{CPS} as the subject of systems to analyse, the entirety of data analysis on the vehicles will create the foundation of a security baseline to protect proactively from external or internal attacks. \par
\stopsection

% Total number of words per section: 400/400

\startsection[title=Existing Models on Data Analysis]
% Relevancy
This section is looking at the existing models that are available and used in the industry that can help to analyse the motion patterns of connected cars. Most of the work has been done on network traffic log analysis and this will be looked at for adoption for interpreting sensor readings. \par
There are several different data analysis methods \cite[authoryear][miles2013qualitative]. Pattern matching tries to find common structure to compare and group against. Sequential and qualitative analysis utilises ordering and takes an analytical approach on the data, whereas sampling will focus on bounding the collection of data. \par
Most of the historical work was conducted on \infull{IDPS}s and the analysis on network log traffic \cite[authoryear][Krugel:2002:SSA:508791.508835]. Although it is certainly a knowledge-base to be built from, the fast changing constellation of connected systems needs to adapt a new approach on dynamic analysis. \par
One of the more recently adopted techniques compromising behavioural data analysis is based on recursive Bayesian estimation \cite[authoryear][Bergman99recursivebayesian], arguing that signature based machine learning of known-good/known-bad states takes too much time to establish a baseline and is prone to multiple false-positive findings. Machine learning tends to be less able to react to real events at an adequate time-frame \cite[authoryear][alpaydin2014introduction] and makes it difficult to react proactively, especially on novel environmental set-ups. \par
By looking specifically on how profiling and behavioural analysis has been analysed on different types of traffic accident scenarios \cite[authoryear][Parker1995571], a good profiling pattern can be derived on the data being processed. \par
% Work done
This paper stresses the significant difference between Bayesian estimation techniques of behavioural analysis and commonly used signature based machine learning solutions in current semi-autonomous cars. \par
% Results found
The results underline that approaching data analysis with predictive behavioural profiling on estimated future states is the better approach for \infull{CPS}. \par
\stopsection

% Total number of words per section: 287/400

\startsection[title=Data Capturing for Connected Things]
% Relevancy
Without valid and quality data that will be captured through the various sensors of the connected car, the analysis would be meaningless. Therefore, it is utmost important to define what data needs to be captured, through what means (readings of sensor data) and that the quality of reading is ensured, coupled with minimal misreading and errors. Some of the inevitably present abnormal readings left in the data-set can be filtered out with mathematical models. \par
For the definition of the data-set and capturing techniques, this paper is looking at established methods already in used within the aviation industry. The global tracking system \infull{ADS-B} (ADS-B) defines prerequisite data types that needs to be sent through radar capturing facilities \cite[authoryear][McCallie201178]. Among the required fields are flight ID, geolocation, time, speed and various other sensor readings. For an illustration of the raw \infull{CSV} (CSV) data, see \in{Table}[table:ads] in the Appendix. The following screenshot of a live ADS-B tracking along an airport gives an indication how this is going to be used in relation to connected cars for this project: \par
\startplacefigure[title={ADS-B Screenshot},list={Automatic Dependent Surveillance - Broadcast},reference=figure:adsb]
 % \externalfigure[http://www.rtl-sdr.com/wp-content/uploads/2013/04/adsbScopeScreenShot1.png][method=png,height=50mm]
 \externalfigure[http://i.imgur.com/CeycqIJ.png][method=png,height=50mm]
\stopplacefigure
More importantly, there is already some form of data capturing available with electric car-sharing providers, the so-called Car-Sharing Modules (CSM) \cite[authoryear][lee2011electric]. Although proprietary, the captured data includes the user renting the car, readings of battery charge, location and speed and log data from the machine learning application. \par
On a higher level, this paper also ponders the viability of \infull{ITS} (ITS), with data from connected vehicles going to a command-and-control centre for intelligent traffic distribution and threat prevention \cite[authoryear][6526427]. \par
% Work done
This paper has proposed and developed a data format for recording vital sensor readings of connected cars, based on the ADS-B model. This open data is suggested as an industry standard for all car manufacturers. \par
% Results found
Having a global standard to compare data and run data analysis on helped creating comparable data sets to show differences in approaches on data analysis. \par
\stopsection

% Total number of words per section: 318/400

\startsection[title=Behavioural Profiling]
% Relevancy
Profiling is the first stage of analysing and concatenating similar groups of behaviour together. Behaviour of like-minded people are similar and drivers in similar roles behave in almost identical patterns. Bus drivers for example have very similar reactions to speed, route and alertness. \par
Different types of individuals but also systems behave differently but deducing and predicting a collection of a group with similar correlation can lead to a conclusion on whether an expected behaviour is unusual. \par
Creating profiles to detect patterns usually assigned to intruders breaking into the system \cite[authoryear][lunt1993detecting] can help determine likelihoods of deviation and raise suspicion on anomalies. Looking specifically at these patterns for a system based profiling as compared to human behaviour profiling of \infull{CPS} is adding value in combination of the estimation techniques discussed in the previous section. \par
Network traffic behaviour has been researched and different profiling data groups have been established \cite[authoryear][4455451], which will be taken advantage of to create parallel profiling templates for connected cars. \par
% Work done
There has been done some research on human behaviour profiling, with different types of groups behaving in a similar pattern (commuting people, families and various other types like taxi drivers). \par
In this paper the focus shifted towards whether a profile pattern can be established with autonomous systems in order to combine these groups with the machine learning estimation techniques. \par
% Results found
It has been shown that indeed it is even easier to group pre-defined system behaviour on predicted routes as compared to humans driving freely on the motorway. Through profiling it is now also possible to distinguish between autonomous and non-autonomous driven vehicles. \par
\stopsection

% Total number of words per section: 262/400

\startsection[title=Anomaly Detection,reference=section:anomalydetection]
% Relevancy
Anomaly detection plays a crucial part in the proactive approach of detecting cyber threats. Various detection techniques (\in{Figure}[figure:anomaly]) are applied to a multitude of scenarios \cite[authoryear][Patcha20073448] to show deviations from the norm through mean functions hitting a threshold and hence indicating suspicious activities. \par

\startplacefigure[title={A summary of statistical anomaly detection systems \\ © An overview of anomaly detection techniques},list={Anomaly Detection Methods},reference=figure:anomaly]
\startxtable[frame=off,option=stretch]
\startxrow[topframe=on,bottomframe=on,foregroundstyle={\tfxx}]
\startxcell[width=23mm] Reference \stopxcell
\startxcell Highlighting feature \stopxcell
\startxcell[width=40mm] Methodology \stopxcell
\stopxrow
\startxrow[foregroundstyle={\tfxx}]
\startxcell Haystack [\startcolor[blue]11\stopcolor] \stopxcell
\startxcell It uses descriptive statistics to model user behavior Also modeled acceptable behavior of a generic user within a particular user group \stopxcell
\startxcell Host based statistical anomaly detection \stopxcell
\stopxrow
\startxrow[foregroundstyle={\tfxx}]
\startxcell NIDES [\startcolor[blue]15\stopcolor] \stopxcell
\startxcell A distributed intrusion detection system that had both anomaly as well as signature detection modules \stopxcell
\startxcell Network based statistical anomaly detection \stopxcell
\stopxrow
\startxrow[foregroundstyle={\tfxx}]
\startxcell Staniford \\ \ \hspace[big] et al. [\startcolor[blue]17\stopcolor] \stopxcell
\startxcell Statistical anomaly detection technique that calculates an anomaly score for each packet that is sees; it forwards the packets to a correlation engine for intrusion detection purposes when a predefined threshold was crossed \stopxcell
\startxcell Network based statistical anomaly detection \stopxcell
\stopxrow
\startxrow[bottomframe=on,foregroundstyle={\tfxx}]
\startxcell Ye et al. [\startcolor[blue]19\stopcolor] \stopxcell
\startxcell It uses the Hotellings {\it T\high{2}} test to analyze the audit trails of activities in an computer system and detect host-based intrusions \stopxcell
\startxcell Host based multivariate statistical anomaly detection \stopxcell
\stopxrow
\stopxtable
% \externalfigure[https://ai2-s2-public.s3.amazonaws.com/figures/2016-11-08/7c3870e4c6f75aefeb7801c75026cefb512304f6/8-Table1-1.png][width=150mm]
\stopplacefigure

Current \infull{IDPS}s (\in{Figure}[figure:ids]) are predominantly using signature-based detecting algorithms of machine learning with rule-based updates to detect known attacks \cite[authoryear][GarciaTeodoro200918], as incident reporting is skyrocketing (\in{Figure}[figure:growth]). \par

\blank[line]

\startcolumns[n=2,rule=on]
\startplacefigure[title={Growth rate of cyber incidents reported to Computer Emergency Response Team/Coordination Center (CERT/CC) \\ \tfx © Intrusion Detection: A Survey},list={Growth Rate of Cyber Incidents},reference=figure:growth]
 % https://www.researchgate.net/profile/Ali_Movaghar/publication/232623012/figure/fig1/AS:341633552928771@1458463194712/Figure-2-1-Growth-rate-of-cyber-incidents-reported-to-Computer-Emergency-Response.png
 \externalfigure[http://i.imgur.com/oZnfHPK.png][method=png,height=35mm]
\stopplacefigure
\column
\startplacefigure[title={Basic architecture of intrusion detection system (IDS) \\ \tfx © Intrusion Detection: A Survey},list={Intrusion Detection System (IDS)},reference=figure:ids]
 % https://www.researchgate.net/profile/Ali_Movaghar/publication/232623012/figure/fig2/AS:341633552928772@1458463194728/Figure-2-4-Basic-architecture-of-intrusion-detection-system-IDS.png
 \externalfigure[http://i.imgur.com/AhIJOms.png][method=png,height=35mm]
\stopplacefigure
\stopcolumns

\blank[line]

Modern trends can be observed however, with a shift towards a more flexible and adaptive anomaly detection mechanisms in contrast to the signature based solutions. This gives a greater security oriented approach on detecting malicious activities. On addition, these methods tend to have also fewer false-positive alarms. \par
Networks are not immune towards intrusions either and studies show \cite[authoryear][doi:10.1137/1.9781611972733.3] that by utilising anomaly detection schemes reaction times can be significantly lowered. \par
This new approach of real-time specification based correlation \cite[authoryear][Sekar:2002:SAD:586110.586146] will give a boost of the "time to detect" factor on malicious behaviour. \par
Another aspect in regards to connected cars are ad-hoc networks and the intrusion detection of such networks. It has been shown that anomaly detection methods are particularly advantageous in such circumstances \cite[authoryear][Zhang:2000:IDW:345910.345958]. \par
% Work done
The novel approach in cyber threat detection on connected cars in this paper has been advanced through getting rid of signature based detection mechanisms and replace it with anomaly detection algorithms that are more purposeful in the connected wireless transportation landscape. \par
% Results found
Results in an increased window to react on malicious activities within the network and a significantly reduced number of false incidents proves that this new approach on cyber risk is very effective indeed. \par

% "An overview of anomaly detection techniques: Existing solutions and latest technological trends"
% "Anomaly-based network intrusion detection: Techniques, systems and challenges"
% "A Comparative Study of Anomaly Detection Schemes in Network Intrusion Detection"
% "Specification-based anomaly detection: a new approach for detecting network intrusions"
% "Intrusion detection in wireless ad-hoc networks"

% http://ieeexplore.ieee.org/abstract/document/5971980/
% http://ieeexplore.ieee.org/abstract/document/1212675/
\stopsection

% Total number of words per section: 269/400

\startsection[title={Threat Prediction Methods},reference=section:threatmodel]
% Relevancy
Before the attention can be shifted toward the advancements in cyber thread detection of connected cars the plethora of prediction methods has to be considered for applicability and suitability detecting malicious activities. \par

\startplacefigure[title={Insider Threat Prediction Model \\ \tfx © An Insider Threat Prediction Model},list={Threat Prediction Model},reference=figure:threatprediction,location={right}]
 % https://www.researchgate.net/publication/220855345/figure/fig1/AS:297441258688512@1447926930054/Fig-1-Insider-threat-prediction-model.png
 \externalfigure[http://i.imgur.com/K8zDVeA.png][method=png,height=35mm]
\stopplacefigure

Taking the cyber security definition of threats from the cyberspace \cite[authoryear][citeulike:13779694] as the base to analyse data traffic will allow the creation of specific attack vectors \cite[authoryear][CPLX:CPLX20001] and the patterns for threat models with targeted attack scenarios. \par
Insider threat prediction \cite[authoryear][Kandias2010] is an interesting angle of how the behaviour of malicious intent tries to blend with legitimate actions (\in{Figure}[figure:threatprediction]). Looking into another field taken from the mathematical models in game theory with a mesh of situational awareness \cite[authoryear][4085956] it can be learned that improvements in threat prediction could also be achieved with a more abstract view. \par
Integrating different prediction methods into a framework for security \cite[authoryear][Chien2012] allows for applying a proactive approach with the specific use case of remote cyber hacking, demonstrated by the action of hijacking. \par
The most interesting predictive models are based on the Recursive Bayesian estimation, also known as Bayes filters. Diving deeper into the mathematical functions, the attention of focus will concentrate on these techniques as they are the most applicable to connected cars.
Whilst in robotics these techniques are commonly used, the extensions of the Bayesian estimation, named sequential Bayesian filtering and made up of filtering, smoothing and prediction, is of particular focus for this paper. \par
% Work done
This paper has applied the sequential Bayesian filtering methods on the specific simulations with connected cars and compared the results in contrast to the machine learning models.
% Results found
It turns out that with the big data created through the sensors whilst in transit it is in an order of magnitude more precise in creating a baseline to compare against and hence these methods are more appropriate for cyber threat prediction. \par

% https://link.springer.com/chapter/10.1007/978-3-642-15152-1_3
% "An Insider Threat Prediction Model" \in{Figure}[figure:threatprediction] \cite[authoryear][Kandias2010]
% http://ieeexplore.ieee.org/abstract/document/4085956/
% "Game Theoretic Approach to Threat Prediction and Situation Awareness" \cite[authoryear][4085956]
% http://onlinelibrary.wiley.com/doi/10.1002/cplx.20001/full
% "Attack scenario graphs for computer network threat analysis and prediction" \cite[authoryear][CPLX:CPLX20001]
% https://link.springer.com/chapter/10.1007/978-3-642-26001-8_1
% "A Novel Threat Prediction Framework for Network Security" \cite[authoryear][Chien2012]
\stopsection

% Total number of words per section: 298/400

\blank[line]

This chapter has looked in-depth at the literature available in regards to the papers research question on proactive threat detection methods with the use case of connected cars and critically evaluated the content. In the next chapter the design of the project, including the experiments and case study can be found. \par

\stopchapter

% The hypothesis or research question and research aim should be the natural conclusion of the literature review.
% Why your chosen question is worth asking (and answering).

% Total number of words: 2072/3000

\stopcomponent
