\startcomponent intro
\product dissertation

\setuphead[chapter][textstyle={\tfd\sc}]

% This chapter is normally the last thing to be written and summarises the entire project.
% Someone reading this introduction would not, in most cases, need to read the rest of the dissertation unless they were interested in specific aspects.

\startchapter[title=Introduction]
The world is getting more and more connected with each other through wireless communication channels and the \infull{IoT} (IoT). There is an increased drive to advance new semi and indeed fully autonomous \infull{CPS} (CPS) that are taking over and shape the traffic of the public space from automotive transportation through aerial vehicles of drones and aircraft systems. New \infull{MANET}s (MANET) and \infull{VANET}s (VANET) are being utilised to exchange and analyse traffic information and steer vehicles through new \infull{ITS} (ITS). \par
The rationale for this paper was motivated by the challenges of increasing cyber threats posed by and to \infull{CPS} \cite[authoryear][loukas2015cyber], specifically, the gap of resilience in addressing the security design of such systems and the lack of real-time cyber threat detection of malicious intruders. \par
Advancements in modern machine learning and algorithms for profiling and behavioural analysis to aid decision making are already widely used, but current implementations are narrow focused on reactive, signature based models. With the knowledge of intrusion prevention systems and anomaly detection through big data analysis on perimeter security, known-good behaviour can be deduced at large-scale movement flows of regulated traffic \cite[authoryear][perallos2015intelligent]. \par
Malfunction of sensors and rogue behaviour \cite[authoryear][graham2016cyber] has never been managed by automated vehicles, which is essential for the Level 4/5 classification of autonomous systems by the \infull{SAE} (SAE) standard and needs to be supported with redundancy and fail-safe mechanisms. \par

\blank[line]

This chapter addresses the problem statement of the research followed by the aims of the project.
Once the objectives this paper wants to achieve are clear, the key findings are presented alongside the elaboration of the contribution to the wider community.
Finally, the closing of this chapter will be laid out with the structure and overview of the entire report. \par

% https://www.researchgate.net/post/Are_there_any_rules_for_using_tenses_in_scientific_papers
% http://academia.stackexchange.com/questions/19268/should-definition-of-terms-section-be-included-in-the-introduction-or-the-litera

% Total number of words per section: 296/300

\setuphead[section][numberstyle={tfb},textstyle={tfb}]
\startsection[title=Problem Statement]
In this section, set out the main aim(s) of your project and explain what they mean. The aim is the fundamental purpose of what you were trying to do – to find out something, to make something, to test something. There is usually only one or two aims and they are often expressed in terms of a hypothesis or research question. \par
\stopsection

% Total number of words per section: 300

\startsection[title=Aim]
% In this section, set out the main aim(s) of your project and explain what they mean. The aim is the fundamental purpose of what you were trying to do – to find out something, to make something, to test something. There is usually only one or two aims and they are often expressed in terms of a hypothesis or research question. \par
The purpose of this study is to demonstrate the pro-active approach of predicting cyber security threats in connected cars, as an example of a cyber-physical system, through deviations in behaviour profiling patterns on location awareness using mathematical probability density functions. \par
\blank[line]
By detecting deviations from predicted desired future state spectrum the project will achieve to create a warning mechanism for cyber threats on cyber-physical systems and being able to react in a fail-safe manner. \par
Introducing a novel approach on proactive
Hence, this research is looking at the current known-good/known-bad methods and related efforts in profiling and behavioural analysis of connected cars used in detecting cyber threats, their weaknesses and showing a novel way of \par
proactive vs reactive methods, good-bad behaviour (cite)
looking at a novel way of predicting cyber threats of a specific sub-set of said cyber-physical systems
\cite[authoryear][citeulike:13779694], by applying a specialised version of the Bayesian estimation to compare the predicted behaviour of cyber-physical systems with the actual motion patterns \cite[authoryear][Fox03bayesianfiltering].
Learning from in / other systems / Taking advantage of
predictive, behavioural
anomalies, deviations
the proposal, resulting, the sections of, whilst, forming
are playing a bigger role than ever
bayesian (to analyse and predict future states)
malicious intent
use case of hijacked connected cars
kill chain?
\stopsection

% Total number of words per section: 300

%\startsection[title=Background]
%This should explain the background and motivation of the project: why did you do it, and why should the reader be interested in what you did. \par
%Definitions: CPS, Connected Cars (Semi-Autonomous), Bayesian, Threat \par
%\stopsection

\startsection[title=Objectives]
%In this section, explain the objectives of your project. These are the sub-goals you needed to achieve in order to complete your project. There will be several objectives – typically around eight. \par
%An objective should be a definite statement that can be clearly identified as "complete" or "not complete". A statement like "To improve my understanding of programming" is too vague, as it is not clear whether it is complete or not. A statement like "To learn how to use dynamic memory structures in C++" is clearer. \par
%You may also wish to include objectives that discuss the parts of your dissertation, like "To produce a literature survey on the subject of…" or "To write a discussion of the implementation of the system". \par
%It is fine to include objectives that you did not complete. Explain why you did not achieve them. Also make clear any new objectives that you have added since writing your research proposal. \par
%http://dissertationwriting.com/introduction-chapter-writing/
In order to achieve the aims and validate the hypothesis the following steps have to be completed:

\startitemize[joinedup,nowhite]
\item Define the criteria of vehicles to be counting as a cyber-physical systems in the study (connected cars)
\item Create a set of data and format on behaviour profiling that will be needed for the analysis and estimation: \color[black]{\it speed, location, time, weight}
\item Compare and contrast different density functions, particularly Bayesian estimation algorithms
\item Combine the estimation techniques to create a formula and algorithm to predict future positions
\item Validate the predicted values on historical data (sensors, simulations)
\item Propose a live warning alarm system (similar to the ground proximity alarm in planes) and a live tracking site (see flightradar24.com)
\item Suggest fail-safe actions to be implemented in autonomous systems (disconnect and safely stop)
\stopitemize

Given these steps, there should be possible to show the holistic approach of anomaly detection in connected cars. \par
\stopsection

% Total number of words per section: 300

\startsection[title=Key Findings]
Work Done and Results \\ Here you should explain the research you carried out, the methodology chosen and the reason for that choice, and the main results and conclusions drawn from those results. \par
short summary of development (concepts,hypothesis,objectives)
\stopsection

% Total number of words per section: 300

\startsection[title=Contribution]
Here you should explain the research you carried out, the methodology chosen and the reason for that choice, and the main results and conclusions drawn from those results. \par
\stopsection

% Total number of words per section: 300

\startsection[title=Structure of the Report]
Here you should {\it briefly} explain the structure of the report so that the reader knows what issues are discussed where. \par
outline for the rest of paper
\stopsection

% Total number of words per section: 300

Before looking at the literature ...
The next chapter will look at the methodology!

\stopchapter

% Total number of words: 2000

\stopcomponent
