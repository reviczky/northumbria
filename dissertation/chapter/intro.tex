\startcomponent intro
\product dissertation

\setuphead[chapter][textstyle={\tfd\sc}]

% This chapter is normally the last thing to be written and summarises the entire project.
% Someone reading this introduction would not, in most cases, need to read the rest of the dissertation unless they were interested in specific aspects.

\startchapter[title=Introduction]
The world is getting more and more connected with each other through wireless communication channels and the \infull{IoT} (IoT). There is an increased drive to advance new semi and indeed fully autonomous \infull{CPS} (CPS) that are taking over and shape the traffic of the public space from automotive transportation through marine vessels, aerial vehicles of drones and aircraft systems. New \infull{MANET}s (MANET) and \infull{VANET}s (VANET) are being utilised to exchange and analyse traffic information and steer vehicles through new \infull{ITS} (ITS). \par
The rationale for this paper was motivated by the challenges of increasing cyber threats posed by and to \infull{CPS} \cite[authoryear][loukas2015cyber], specifically, the gap of resilience in addressing the security design of such systems and the lack of real-time cyber threat detection of malicious intruders. \par
Advancements in modern machine learning and algorithms for profiling and behavioural analysis to aid decision making are already widely used, but current implementations are narrow-focused on reactive, signature based models. With the knowledge of intrusion prevention systems and anomaly detection through big data analysis on perimeter security, known-good behaviour can be deduced at large-scale movement flows of regulated traffic \cite[authoryear][perallos2015intelligent]. \par
Malfunction of sensors and rogue behaviour \cite[authoryear][graham2016cyber] has never been managed by automated vehicles, which is essential for the Level 4/5 classification of autonomous systems by the \infull{SAE} (SAE) standard and needs to be supported with redundancy and fail-safe mechanisms. \par

\blank[line]

This chapter addresses the problem statement of the research followed by the aims of the project.
Once the objectives this paper wants to achieve are clear, the key findings are presented alongside the elaboration of the contribution to the wider community.
Finally, the closing of this chapter will be laid out with the structure and overview of the entire report. \par

% https://www.researchgate.net/post/Are_there_any_rules_for_using_tenses_in_scientific_papers
% http://academia.stackexchange.com/questions/19268/should-definition-of-terms-section-be-included-in-the-introduction-or-the-litera

% Total number of words per section: 291/300

\setuphead[section][numberstyle={tfb},textstyle={tfb}]
\startsection[title={Problem Statement},reference=section:problemstatement]
Currently, the problem of resilience for cyber-attacks on autonomous \infull{CPS} is not tackled. Weaknesses in semi-autonomous (autopilot) cars have been already shown through local exploitation \footnote{\goto{www.bloomberg.com/features/2015-george-hotz-self-driving-car}[url(https://www.bloomberg.com/features/2015-george-hotz-self-driving-car/)]} and interconnecting these systems will lead to an even bigger attack surface. Whilst the industry is moving ahead with big steps to produce self-driving cars, buses and trains, autonomous flying drones and various other transportation systems, the aspect of built-in security is falling short. Moreover, the \infull{IoT} creates a world where devices and systems can be taken over and cause significant disruptions in our day-to-day lives. \par
This leaves an open door to targeted attacks and malicious intent starting from small criminals to state sponsored attacks to harm individuals or gain influence through sabotage and even industry espionage \cite[authoryear][6459914].
With over a billion cars owned in the world and being potentially replaced with new, connected vehicles this papers sets out to show how to solve this deficiency through defining behavioural analysis of such systems with the help of profiling by looking particularly at autonomous connected cars. Profiling will help addressing the correlation of various systems, whereas through behavioural analysis it can be deducted whether an expected state of a sensor reading is within the margins of errors. A novel method to detect and react to a specific use case of cyber hijacking will be presented in order to prove that effective pro-active security can be built-in and create resilience towards cyber-attacks. \par
This technique shown through the geo-data with the scenario of a hijacking can be used further by introducing the method in combination with other sensor data readings. Current implementations of smart systems which are utilising machine learning to detect anomalies \cite[authoryear][5166473] are impractical for vehicles. \par
\stopsection

% Total number of words per section: 281/300

\startsection[title=Aim,reference=section:aim]
% In this section, set out the main aim(s) of your project and explain what they mean.
% The aim is the fundamental purpose of what you were trying to do – to find out something, to make something, to test something.
% There is usually only one or two aims and they are often expressed in terms of a hypothesis or research question.

The aim of this project is to demonstrate that by incorporating different applications of Bayesian estimation techniques a novel method can be developed and established to counter cyber-attacks. With the specific scenario of a cyber hijacking, given a connected car driving between two cities, it can be shown that pro-active resiliency is addressed and that cyber defence can be achieved. \par
Once it is proven that behavioural analysis is superior to signature based machine learning for real-time predictive modelling of future states for comparable travel routes and actual motion patterns \cite[authoryear][Fox03bayesianfiltering], in terms of the tolerance on the deviation from the norm of behavioural profiling patterns on location awareness using mathematical probability density functions, anomaly detection can be reacted upon far quicker and more accurately than before. This research is looking at the current related efforts of methods in profiling and behavioural analysis \cite[authoryear][alpaydin2014introduction] and their weaknesses. Hence, the fundamental purpose of this paper is to show the comparison of time-spans to react to cyber-attacks between current industry-standard machine learning algorithms of known-good/known-bad behaviour and the new approach of behavioural analysis through a specialised version of the Bayesian estimation and proving the superiority of the latter. \par
With the significant improvements to reaction times, malicious intent can be countered to eliminate the intended harm and a pro-active stance can be taken instead of the current re-active solutions to predict cyber security threats as early as they happen in the kill chain. \par
To prove that this new approach of behavioural analysis can be the baseline of cyber defence of \infull{CPS} \cite[authoryear][citeulike:13779694], this paper will demonstrate the technique on a specific scenario of connected cars (vehicles) as an example, from which it can be deduced that the technique is fit for purpose in the general application of land-based or aerial transportation.
By detecting deviations from a predicted desired future state spectrum, the project will achieve to create the fundamentals of a warning mechanism for cyber threats on \infull{CPS} and the ability to react in a fail-safe manner. \par
\stopsection

% Total number of words per section: 332/300

%\startsection[title=Background]
% This should explain the background and motivation of the project: why did you do it, and why should the reader be interested in what you did.
%\stopsection

\startsection[title=Objectives,reference=section:objectives]
% In this section, explain the objectives of your project.
% These are the sub-goals you needed to achieve in order to complete your project.
% There will be several objectives – typically around eight.
% An objective should be a definite statement that can be clearly identified as "complete" or "not complete".
% A statement like "To improve my understanding of programming" is too vague, as it is not clear whether it is complete or not.
% A statement like "To learn how to use dynamic memory structures in C++" is clearer.
% You may also wish to include objectives that discuss the parts of your dissertation, like "To produce a literature survey on the subject of…" or "To write a discussion of the implementation of the system".
% It is fine to include objectives that you did not complete.
% Explain why you did not achieve them.
% Also make clear any new objectives that you have added since writing your research proposal.

% http://dissertationwriting.com/introduction-chapter-writing/

In order to achieve the aims and validate the hypothesis the following steps have to be completed to deliver the project objectives: \par

\blank[line]

\startitemize[joinedup,nowhite]
\sym{»} To define the sub-set and criteria of autonomous \infull{CPS} that are described as connected cars (vehicles) in the examples of the case study
\sym{»} To create a scenario with a set of data \color[black]{\tf (direction, speed, location, time)} and proposing a data format that will be needed for the analysis and estimation on behavioural profiling
\sym{»} To compare and contrast different density functions, particularly Bayesian estimation algorithms to determine the suitability for real-time anomaly detection
\sym{»} To combine the estimation techniques to create a formula and algorithm for predicting future states (positions)
\sym{»} To validate the predicted values on historical data (sensor data) on pre-defined simulations
\sym{»} To suggest fail-safe actions to be implemented in autonomous systems (disconnection and stopping safely) in the event of a breach
\sym{»} To show the improvements that can be achieved with behavioural analysis in comparison to signature based machine learning
\sym{»} To propose a live warning alarm system (similar to the ground proximity alarm in aeroplanes) and a live tracking site (see flightradar24 \footnote{\goto{flightradar24.com}[url(https://www.flightradar24.com/)]}) to form a control centre against cyber-attacks
\stopitemize

\blank[line]

Given the completion of these goals, it is possible to show how a holistic approach of anomaly detection in connected cars can be incorporated into the security design to limit the effects of a cyber-attack. \par
Follow-up objectives can be defined in the application of the presented techniques, especially on the ethical aspects of reaction based decisions, but these do not form part of this project. \par
\stopsection

% Total number of words per section: 257/300

\startsection[title=Key Findings]
% Here you should explain the research you carried out, the methodology chosen and the reason for that choice, and the main results and conclusions drawn from those results.
% Development of concept, hypothesis and objectives.

This paper has developed and presented a novel way of cyber threat prevention by using concepts of behavioural analysis through Bayesian estimation techniques and the following work has been carried out with the results to prove the method significantly improves accuracy and the time to detect malicious intent. \par
Connected cars have been chosen as a use case for the research to focus on a sub-set of \infull{CPS} and conduct the behavioural analysis with a specific scenario of cyber hijacking. In order to prove the hypothesis a quantitative research design has been utilised by sampling a data set of routes between two cities and the motion patterns including sensor data on which the various methods and techniques are then applied. Through the sampling it can be shown that the deviations from normal routes can be recognised far more in advance than current industry-standard methods. \par
A new method is presented with a combination of profiling and behavioural analysis that is conducted on the sampled data for detecting anomalies and predicting future states. To supplement the cyber defence a new fail-safe function of connected systems is also presented. The main results of the research includes an increased window for time to react in real-time. It also shows that accuracy through profiling is vastly increased and false-positives are reduced significantly. Comparing and showing that signature based machine learning is inferior to behavioural analysis also proves that cyber defence on transportation is not tackled with the right techniques. \par
Ultimately, the paper concludes that incorporating Bayesian estimation techniques can counter cyber threats with the mind-set of a pro-active approach instead of current re-active systems in a more effective way. \par
\stopsection

% Total number of words per section: 274/300

\startsection[title=Contribution]
% Contribution to the community.

Whilst the increasing number of autonomous systems are playing a bigger role than ever, the need of a new framework taking cyber security design into the architecture of the devices to protect against the increasing vector of cyber-attacks is almost a necessity. \par
This paper is presenting a security baseline for \infull{CPS} that can be incorporated into various different transportation vehicles, hence, giving protection to the passenger against cyber-attacks and therefore making the transportation safer. By showing a new angle on how to tackle local defence in real-time and contained on the object itself, the advantages and results can be utilised to create more intelligent and fail-safe automation. Further, the same logic could be applied to a command-and-control system with \infull{ITS}. \par
Learning from the evolution of \infull{IDPS} and taking advantage of machine learning based artificial intelligence solutions creates a knowledge base on which to build on and gives input on how to improve detection of malicious intruders. Making the groundwork for system profiling and anomaly detection based on behavioural analysis will be very valuable when looking at the next generation of transportation methods. \par
Finally, these findings create a source of behavioural data that the community can take away to create a holistic warning system which could track cyber threats and connect the incidents to deduce the scale and the target of cyber threats which in turn can be then looked at from a cyber risk perspective and acted upon accordingly. \par
\stopsection

% Total number of words per section: 246/300

\startsection[title=Structure of the Report]
% Here you should {\it briefly} explain the structure of the report so that the reader knows what issues are discussed where.

The outline of this research paper is structured in the following manner: \par
The leading thread aligns to the logical frame of chapters and sections within, starting with explaining the methodology of the research that has been carried out (\in{Chapter}[chapter:methodology]). Continuing with the literature review (\in{Chapter}[chapter:literature]) the design (\in{Chapter}[chapter:design]), results (\in{Chapter}[chapter:result]) and discussions (\in{Chapter}[chapter:discussion]) will form the main part of the dissertation. The last chapter will conclude with the findings (\in{Chapter}[chapter:conclusion]) and achievements of the aims, including the objectives that have been defined and met in this chapter. Further work on the outcome of this project will be indicated, which will give a closure of this work. \par
Each chapter will start with a short introduction of what issue the chapter is discussing and being closed with a brief summary and referencing the next steps of the successive chapter. \par
Excluding the excerpts, all referred graphics, figures and tables are included in their respective sections within the appendix (\in{Appendix}[appendix:figures]) and full references are being given. Further, all of the abbreviations used within the paper are written out in full at the end of this paper. \par
The source code for each simulated example is collected and presented in the appendix (\in{Appendix}[appendix:code]) in order to reproduce the findings and are licensed as GNU General Public License v2.0 (\goto{GNU GPL v2.0}[url(https://www.gnu.org/licenses/old-licenses/gpl-2.0.en.html)]). \par
For every other attachment (including project logs and ethical and risk forms) that is set out by the University of Northumbria, the appropriate documents are provided in the appendix section of the paper as required. \par
\stopsection

% Total number of words per section: 252/300

\blank[line]

This concludes the introduction chapter, where the issues of background and motivation of the project, the problem statement in combination of the aims and objectives were discussed and the key findings alongside the contribution to the community were presented and closed off with the section of the structure of this report. \par
The next chapter will look at the methodology that is used in this paper, before the review of the literature is undertaken. \par

\stopchapter

% Total number of words: 2006/2000

\stopcomponent
