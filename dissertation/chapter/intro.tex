\startcomponent intro
\product dissertation

\setuphead[chapter][textstyle={\tfd\sc}]
\startchapter[title=Introduction]
This chapter is normally the last thing to be written and summarises the entire project. Someone reading this introduction would not, in most cases, need to read the rest of the dissertation unless they were interested in specific aspects. \par

\setuphead[section][numberstyle={tfb},textstyle={tfb}]
\startsection[title=Aims]
In this section, set out the main aim(s) of your project and explain what they mean. The aim is the fundamental purpose of what you were trying to do – to find out something, to make something, to test something. There is usually only one or two aims and they are often expressed in terms of a hypothesis or research question. \par
\stopsection

\startsection[title=Background]
This should explain the background and motivation of the project: why did you do it, and why should the reader be interested in what you did. \par
\stopsection

\startsection[title=Objectives]
In this section, explain the objectives of your project. These are the sub-goals you needed to achieve in order to complete your project. There will be several objectives – typically around eight. \par
An objective should be a definite statement that can be clearly identified as "complete" or "not complete". A statement like "To improve my understanding of programming" is too vague, as it is not clear whether it is complete or not. A statement like "To learn how to use dynamic memory structures in C++" is clearer. \par
You may also wish to include objectives that discuss the parts of your dissertation, like "To produce a literature survey on the subject of…" or "To write a discussion of the implementation of the system". \par
It is fine to include objectives that you did not complete. Explain why you did not achieve them. Also make clear any new objectives that you have added since writing your research proposal. \par
\stopsection

\startsection[title=Work Done and Results]
Here you should explain the research you carried out, the methodology chosen and the reason for that choice, and the main results and conclusions drawn from those results. \par
\stopsection

\startsection[title=Structure of the Report]
Here you should {\it briefly} explain the structure of the report so that the reader knows what issues are discussed where. \par
\stopsection
\stopchapter

\stopcomponent
