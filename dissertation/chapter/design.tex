\startcomponent design
\product dissertation

\usemodule[chart]
\setupFLOWcharts[nx=4,ny=2,dx=2\bodyfontsize,dy=2\bodyfontsize,maxwidth=\textwidth,height=3\lineheight]
\setupFLOWshapes[framecolor=black,background=color,backgroundcolor=white]

\setuphead[chapter][textstyle={\tfd\sc}]
\startchapter[title=Design,reference=chapter:design]

% This should discuss any practical work done.
% If a significant software product is built then this should include a discussion of each stage of the software development life cycle, including:
% Definition of requirements; Design; Implementation; Testing
% Along with consideration of any tools and technologies used (languages, IDEs, third-party components used, etc).
% Similarly if your practical work has consisted of experimental work, then this should include a discussion of each stage of the experimental process including
% Definition of experimental variables and other requirements; Design of experiments; Implementation; Collection of Results
% Along with consideration of any tools and technologies used.
% This discussion should outline the choices made, along with any alternatives considered and the reasons for those choices.

% This section should assess the discussion of the practical work you have done, such as requirements analysis, software design, construction, installation, experimental work.
% Your testing and/or data collection approach should be included in this section although the results from this form part of the next section.
% You should make it clear what you have done, but you should also include rationales for the approaches and techniques used, as well as a discussion of any ideas that have been rejected.

Any practical or significant work that has been done is going to be described and discussed in this chapter, which includes the requirements, design and planning phases. \par
A detailed illustration of the experimental case study will be given along the whole design of the project. Looking at the supporting software utilised, the requirements, construction and usage will be explained. Each of the tests will be validated and the data collection for such testing shown. \par

\blank[line]

The main idea of the work is based on the probabilistic Recursive Bayesian estimation (also known as Bayes filter) \footnote{\goto{Wikipedia: wikipedia.org/wiki/Recursive Bayesian estimation}[url(https://en.wikipedia.org/wiki/Recursive_Bayesian_estimation)]} technique which will be referred to in all instances. \par
Looking at the future states, the probability of the current given state in relation of the very previous state is conditionally independent and expressed with the following formula: \par

\placenamedformula[formula:markov]{Markov Assumption}
\definereferenceformat[eqref][left=(,right=)]
%\setupnumber[formula][way=bysection]
\setupformulas[location=left]
\startformula
p(\bf{x}_k|\bf{x}_{k-1},\bf{x}_{k-2},\dots,\bf{x}_0) = p(\bf{x}_k|\bf{x}_{k-1})
\stopformula

Where the probability distribution of the states at the \infull{HMM} (HMM) is written as: \par

\placenamedformula[formula:hmm]{Hidden Markov Model (HMM)}
\startformula
% p(\bf{z}_k|\bf{x}_k,\bf{x}_{k-1},\dots,\bf{x}_{0}) = p(\bf{z}_k|\bf{x}_{k})
p(\bf{x}_0,\dots,\bf{x}_k,\bf{z}_1,\dots,\bf{z}_k) = p(\bf{x}_0)\prod_{i=1}^k p(\bf{z}_i|\bf{x}_i)p(\bf{x}_i|\bf{x}_{i-1})
\stopformula

The following figure illustrates the flow of a Bayesian Network within a HMM: \par

% \startplacefigure[title={Kalman Filter \\ \tfx © Wikipedia},list={Kalman},reference=figure:kalman]
\startplacefigure[title={Hidden Markov Model},list={Hidden Markov Model},reference=figure:markov]
% \externalfigure[https://upload.wikimedia.org/wikipedia/commons/thumb/8/81/HMM_Kalman_Filter_Derivation.svg/466px-HMM_Kalman_Filter_Derivation.svg.png][width=50mm]
\scale[scale=500]{
\startFLOWchart[markov]
\startFLOWcell
 \name{bay0}
 \location{1,1}
 \shape{0}
 \connection[rl]{bay1}
\stopFLOWcell
\startFLOWcell
 \name{bay1}
 \location{2,1}
 \shape{35}
 \text{$$x_{k-1}$$}
 \connection[rl]{bay2}
 \connection[bt]{bay3}
\stopFLOWcell
\startFLOWcell
 \name{bay2}
 \location{3,1}
 \shape{35}
 \text{$$x_k$$}
 \connection[bt]{bay4}
 \connection[rl]{bay99}
\stopFLOWcell
\startFLOWcell
 \name{bay3}
 \location{2,2}
 \shape{24}
 \text{$$z_{k-1}$$}
\stopFLOWcell
\startFLOWcell
 \name{bay4}
 \location{3,2}
 \shape{24}
 \text{$$z_k$$}
\stopFLOWcell
\startFLOWcell
 \name{bay99}
 \location{4,1}
 \shape{0}
\stopFLOWcell
\stopFLOWchart
\FLOWchart[markov]
}
\stopplacefigure

\startsection[title=Experimental Work]
% Definition of experimental variables and other requirements; Design of experiments; Implementation; Collection of Results

% requirements analysis, software design, construction, installation, experimental work
% testing and/or data collection approach
% make it clear what you have done, but you should also include rationales for the approaches and techniques used, as well as a discussion of any ideas that have been rejected

This is the detailed course of action the research took for the experiments conducted. \par

The software catalogue for tools used in the experimental work consist of the following open source applications and libraries: \par
\startitemize[joinedup,nowhite]
\sym{»} PyMC (version 2 and 3) \endash\ {\it is a python module that implements Bayesian statistical models and fitting algorithms, including Markov chain Monte Carlo}
\sym{»} Stan \endash\ {\it statistical modeling, data analysis, and prediction in the social, biological, and physical sciences, engineering, and business}
\sym{»} PyBayes \endash\ {\it Python library for recursive Bayesian estimation (Bayesian filtering)}
\stopitemize

Frameworks and tools that have been used include: \par

\startitemize[joinedup,nowhite]
\sym{»} RStudio (IDE) \endash {\it Open source and enterprise-ready professional software for R} / R language via the R Project for Statistical Computing / GNU Compiler Collection
\sym{»} Python Programming Language
\stopitemize

The entire source code for this project is available on the \goto{github.com/reviczky}[url(https://github.com/reviczky/northumbria)] site with the source code part licensed under the GPLv2, so that every part of these experiments can be re-constructed. \par
Choices made for these tools to support the experiments were based on the availability under open source license \footnote{\goto{opensource.org/osd-annotated}[url(https://opensource.org/osd-annotated)]} in order to have scientific peer-review. There are many alternatives, as shown in the critical review of the literature on the subject, but their choice is irrelevant, as any tool that can prove the theory will suffice. \par
Contemporary commercial approaches have been looked at from Tesla (MobileEye) \footnote{\goto{heise.de/newsticker/meldung/Tesla-Model-S-laesst-sich-von-fern-kapern-3327510.html}[url(http://www.heise.de/newsticker/meldung/Tesla-Model-S-laesst-sich-von-fern-kapern-3327510.html)]} as well as Darktrace (Enterprise Immune System), but given the nature of those being proprietary and patented, the input was merely to put this research into perspective of the current technologies. \par

% "Bayesian Methods for Hackers" \cite[authoryear][Davidson-Pilon:2015:BMH:2851115] \footnote{\goto{Getting our priorities straight}[url(http://nbviewer.jupyter.org/github/CamDavidsonPilon/Probabilistic-Programming-and-Bayesian-Methods-for-Hackers/blob/master/Chapter6_Priorities/Ch6_Priors_PyMC3.ipynb)]} \\
% \goto{Probabilistic}[url(http://camdavidsonpilon.github.io/Probabilistic-Programming-and-Bayesian-Methods-for-Hackers/)] \\
% \goto{CamDavidsonPilon}[url(http://nbviewer.jupyter.org/github/CamDavidsonPilon/Probabilistic-Programming-and-Bayesian-Methods-for-Hackers/blob/master/Chapter6_Priorities/Chapter6.ipynb)] \\
One of the best resources to get an overview of the tools and their utilisation for the experiments can be found in the "Probabilistic Programming and Bayesian Methods for Hackers" \cite[authoryear][Davidson-Pilon:2015:BMH:2851115] \footnote{\goto{camdavidsonpilon.github.io/Probabilistic-Programming-and-Bayesian-Methods-for-Hackers}[url(http://nbviewer.jupyter.org/github/CamDavidsonPilon/Probabilistic-Programming-and-Bayesian-Methods-for-Hackers/blob/master/Chapter1_Introduction/Ch1_Introduction_PyMC3.ipynb)]}. \par
Normal regression with Bayesian estimation using \WORD{stan} is described in the article "Bayesian regression with \WORD{stan}" \footnote{\goto{r-bloggers.com/bayesian-regression-with-stan-part-1-normal-regression}[url(https://www.r-bloggers.com/bayesian-regression-with-stan-part-1-normal-regression/)]}, which is used for the experiments described in \in{Section}[section:casestudy]. \par
Other regression models have been looked at with prior distribution for logistics \cite[authoryear][gelman2008] \footnote{\goto{stat.columbia.edu/~gelman/research/published/priors11.pdf}[url(http://www.stat.columbia.edu/~gelman/research/published/priors11.pdf)]}, but have been ultimately rejected based on the non-specific nature of the model. \par
By using the ShinyStan \footnote{\goto{jsg2201.shinyapps.io/ShinyStanDemo}[url(https://jsg2201.shinyapps.io/ShinyStanDemo/)]} \infull{SaaS} (SaaS) solution to plot the beta functions for density correlation it is possible to do a pairwise correlation as seen in this figure: \par

\startplacefigure[title={Pairwise Correlation},list={Pairwise Correlation},reference=figure:correlation]
 \externalfigure[https://i0.wp.com/datascienceplus.com/wp-content/uploads/2016/01/stan_norm2.png][width=75mm]
\stopplacefigure
\stopsection

% Total number of words per section: 306/250

\startsection[title={Data Collection},reference=section:datacollection]

Planning and data collection consisted of two essential parts that need to be defined before forward processing is possible. Firstly, a collection of sensor readings that could make up the schematics of the open data \footnote{\goto{opendatahandbook.org}[url(http://opendatahandbook.org/)]}. In that regard the following readings have been defined: \par

\startitemize[joinedup,nowhite]
\sym{»} Speed metrics (m/s) through air pressure
\sym{»} Geolocation (longitude and latitude) via GPS positioning
\sym{»} Elevation (altitude) via GPS positioning
\sym{»} Orientation (cardinal direction) via gyroscopes
\sym{»} UTC date-stamps via the \infull{NTP} (NTP)
\sym{»} IPv6 addresses for unique identification of systems (see The Internet Society \footnote{\goto{internetsociety.org/ipv6-frequently-asked-questions}[url(https://www.internetsociety.org/ipv6-frequently-asked-questions\#seventeen)]})
\stopitemize

Secondly, once the data types have been defined, the creation of a database from these metrics have been based on the established format used for broadcasting aeroplanes in the aviation industry via ADB-S, see full \in{Table}[table:ads]. \par
\startplacefigure[title={ADS-B CSV Database},list={ADS-B CSV Database},reference=figure:ads_csv]
% http://cdn-misc.pinkfroot.com/3C6618.csv
\starttyping
adshex,lat,lon,mtime,altitude,heading,speed,vertrate,flightno
3C6618,44.1141,6.32636,1427189963,10275,26.3,409.2,-2560,4U9525
3C6618,44.0445,6.28003,1427189923,12300,26.6,412.8,-3008,4U9525
3C6618,44.0196,6.26355,1427189908,13000,26.1,429.8,-3520,4U9525
3C6618,43.9671,6.22835,1427189879,14700,26.5,428,-3456,4U9525
\stoptyping
\stopplacefigure
Collecting a sample set of data so that the experiments could be analysed for correlation and anomaly detection, two routes have been created (\in{Section}[section:experiments]), based on real route data but artificial motion patterns to avoid ethical conflicts on personal data and safety. The full table is available on github.com \footnote{\goto{github.com/reviczky/northumbria/tree/master/dissertation/database}[url(https://github.com/reviczky/northumbria/tree/master/dissertation/database/)]}. \par
Having established the data sources and the data types it should be noted that the local collection of cached data is important to run on-the-fly data analysis on the big data as well as for redundancy, error correction and not least forensics (for legal reasons). This could be achieved with data recorders, also commonly known as black boxes for connected cars, but a central data collection service needs to be considered as well, especially when thinking about ITS. \par
With the data at hand, the application of modern data science to analyse the data will play a vital role for this project. \par
\stopsection

% Total number of words per section: 274/250

\startsection[title={Estimation Model},reference=section:model]
The aim of the project (\in{Section}[section:aim]) was to choose the right model for predicting future states in order to deduce anomaly behaviour. The literature review (\in{Chapter}[chapter:literature]) highlighted a variety of mathematical models that could be applied on the side of existing models currently used in the transportation industry. With the proposal on settling with Bayesian estimation techniques the focus in this section is on recursive Bayesian estimation (or in short Bayes filter). \par
The following three applications are available for Bayesian estimation on general probabilistic methods: \par

\startitemize[joinedup,nowhite]
\sym{»} Kalman filter
\sym{»} Markov process
\sym{»} \infull{HMM} (HMM), see \in{Formula}{.}[formula:hmm]

\stopitemize

Estimating unknown future states (through incoming metrics) via recursive density functions allow the usage of three types of mathematical models: \par

\startitemize[joinedup,nowhite]
\sym{»} Sequential Bayesian filtering: Filtering (estimate the current value)
\sym{»} Sequential Bayesian filtering: Smoothing (estimating past values)
\sym{»} Sequential Bayesian filtering: Prediction (estimating a probable future value)
\stopitemize

These sequential Bayesian filtering methods, which are extensively used in robotics and other embedded control devices are prefect to be applied on the data collected in the previous section (\in{Section}[section:datacollection]) in order to perform prediction analysis. \par
Plotting density functions with these models is to enable comparisons through correlation for behavioural patterns (algorithms) this project is interested in analysing. \par
The following Bayesian estimation equational expressions are used in libraries for the plotting of the use cases in this project: \par

\placenamedformula[formula:markov]{Markov Process}
\startformula
P(x(t_n) <= x_n | x(t_{(n-1)}),\dots,x(t_1)) = P(x(t_n) <= x_n | x(t_{(n-1)}))
\stopformula

% MathWorld Bayesian Entry
A detailed explanation of the formula is available on the reference work MathWorld \footnote{\goto{mathworld.wolfram.com/MarkovProcess.html}[url(http://mathworld.wolfram.com/MarkovProcess.html)]}. \par

Further, looking at the Kalman filter \footnote{\goto{mathworld.wolfram.com/KalmanFilter.html}[url(http://mathworld.wolfram.com/KalmanFilter.html)]} to optimise imprecise linear or near-linear data is expressed in this formula: \par

\placenamedformula[formula:markov2]{Markov Process Stochastic}
\startformula
P(x(t_n) <= x_n | x(t)\ for\ all\ t <= t_{(n-1)}) = P(x(t_n) <= x_n | x(t_{(n-1)}))
\stopformula

Lastly, looking at probabilistic models using Bayesian methodology in programming for a unifying framework \cite[authoryear][diard:hal-00019254] several filtering and smoothing algorithms \cite[authoryear][Srkk:2013:BFS:2534502] will be incorporated into the method used in this project. \par
\stopsection

% Total number of words per section: 260/250

\startsection[title={Correlation Graphs}]
Utilising the estimation formulas in the previous section (\in{Section}[section:model]) to the specific scenarios introduced at the beginning of this paper (\in{Section}[section:experiments]) correlation analysis can visually represent the baseline and anomalies for the routes taken. \par
The first correlation the experiment will focus on is whether there is a convergence for the expected route taken in terms of future states and how much the deviation of from the baseline this convergence is off in percentage terms. This is represented in the first correlation graph (\in{Figure}[figure:correlation]) below. \par

\startcolumns[n=2,rule=on]
\startplacefigure[title={Convergence of Expected Values \tfx © Cam Davidson-Pilon},list={Convergence of Expected Values},reference=figure:convergence]
 % http://nbviewer.jupyter.org/github/CamDavidsonPilon/Probabilistic-Programming-and-Bayesian-Methods-for-Hackers/blob/master/Chapter4_TheGreatestTheoremNeverTold/Ch4_LawOfLargeNumbers_PyMC2.ipynb
 \externalfigure[http://i.imgur.com/aGFa1XV.png][width=85mm]
\stopplacefigure
\column
\blank[line] \ 
\startplacefigure[title={Empirical Returns \\ \tfx © Cam Davidson-Pilon},list={Empirical Returns},reference=figure:empirical]
 % http://nbviewer.jupyter.org/github/CamDavidsonPilon/Probabilistic-Programming-and-Bayesian-Methods-for-Hackers/blob/master/Chapter5_LossFunctions/Ch5_LossFunctions_PyMC2.ipynb
 \externalfigure[http://i.imgur.com/Nl1jPMN.png][width=75mm]
\stopplacefigure
\stopcolumns

The very next aspect on correlating events is how well the distribution of routes taken will make up a picture on profiling based behavioural analysis. This is shown above in the correlation graph for empirical returns (\in{Figure}[figure:empirical]). \par
The full plotting and results for the scenarios of connected cars can be found in the Appendix part of the paper (see \in{Chart}[chart:rstan] and \in{Chart}[chart:deviation]). \par
One of the subjects of the paper's theory is the creation of motion profiles, divided into behavioural groups of drivers or common systems (think about goods vehicles versus taxis). Each of the correlation graphs building a picture of the profiles the driving group has in common. Even though there is no need to actually define these profiles, as grouping them is enough for applying estimation techniques, but further research could be conducted on the results shown here for the specific group behaviours. \par
Collectively looking at the patterns these correlation graphs create, the deviation function becomes clearer. It has been shown through the analytical work, that this deviation can be ring-fenced to a 5\% margin with the estimation techniques using Bayes filters. A representative chart is shown in the Appendix \in{Chart}[diagram:flowchart]. \par
\stopsection

% Total number of words per section: 283/250

\startsection[title={Case Study: Connected Cars},reference=section:casestudy]
This paper is proving the problem statement and theory on proactive threat detection by the means of a case study with autonomous connected cars. \par
The specific experiments for the two chosen routes are described in the methodology (\in{Section}[section:experiments]). \par
Route 1 is taken with the scenario of a connected car on the "Autobahn" in Germany, whereas the second example is in contrast with the driving side being on the left in the United Kingdom. This way, the results should be not biased or skewed towards patterns that would not be supportive of the theory. \par
Sensors readings were oriented to the ones seen in the semi-autonomous vehicles found in the likes of Tesla and BMW (see \in{Appendix}[appendix:figures] \in{Figure}[figure:tesla]). Sampling on the movements of available routes is also of advantage, as there is a very good set for comparison of machine learning and profile based behavioural analysis. \par
The advantage of an example scenario with connected cars instead of, for example, connected planes or drones is that with vehicles on the road it is also possible to analyse human behaviour in contrast to programmed system movement. \par
The threat modeling on connected cars is represented in this figure: \par

\startplacefigure[title={Connected Car Threat Modeling \\ © Guardtime},list={Threat Modeling},reference=figure:threatmodel]
 \externalfigure[http://f.edicy.com/0000/0036/0235/photos/car_diagram_large.jpg][width=85mm]
\stopplacefigure

Organisation like the Automotive Cyber Security \footnote{\goto{automotivecybersecurity.iqpc.com}[url(https://automotivecybersecurity.iqpc.com/)]} and the Cyber Secure Car \footnote{\goto{cybersecurecar.com}[url(https://www.cybersecurecar.com/europe/)]} enables to engage with the community to advance and present the discussions around cyber threat and resilience findings this paper has presented and can bridge the gap between research and industry. \par
\stopsection

% Total number of words per section: 238/250

\startsection[title=Particular Events]
Particular events and threats that were used in the scenarios for the use case of connected cars are in the nature of cyber risk and the threat landscape of the wider \infull{IoT}. \par
The main risk that could arise from the cyberspace is a remote hack into the \infull{CPS}, in this case the connected car, and to change course of action for the travel in a way to either divert or create damage by driving the vehicle against the rules set out in traffic management and laws that everyone abides. This scenario is described as cyber hijacking of the connected car. \par
Obviously, there are many more threats that could also be posed by state sponsored attacks up to a mass-hijacking of vehicles. Another factor to consider is also system error and pilot error. Any component in the car, especially sensors that are being relied upon can malfunction and give therefore false readings. Proper redundancy as well as fail-over needs to be in place on a hardware level to deal with such events, but also filtering out false events in the logs from a software perspective needs to be thought of when designing the solution. \par
The threat prevention methods are described in detail in \in{Section}[section:threatmodel] and resilience is already used in various forms of machine learning and artificial intelligence for particular applications of threats to counter. \par
\stopsection

% Total number of words per section: 228/250

\startsection[title={Similar Requirements for the Community}]
A very important aspect of this project is to engage with the community in order to match up with similar requirements as well as discussing the findings in the experiments that were presented here. \par
The list of communities for connected cars in the automotive industry to seek engagement with are: \par

\blank[line]

\startitemize[joinedup,nowhite]
\sym{»} Automotive Cyber Security Thought Leadership \footnote{\goto{theiet.org/sectors/transport/documents/automotive-cs.cfm}[url(http://www.theiet.org/sectors/transport/documents/automotive-cs.cfm)]}
\sym{»} \externalfigure[https://iotsecurityfoundation.org/wp-content/uploads/2016/08/IoTSF-website-3.png][height=\lineheight] IoT Security Foundation \footnote{\goto{iotsecurityfoundation.org}[url(https://iotsecurityfoundation.org/)]}
\sym{»} \externalfigure[http://nmi.org.uk/wp-content/uploads/2015/06/nmi-parent-logo-complete-lo.png][height=\lineheight] Connected Communities \footnote{\goto{nmi.org.uk/event/automotive-security-conference}[url(https://nmi.org.uk/event/automotive-security-conference/)]}
\sym{»} \externalfigure[https://www.automotivelinux.org/sites/cpstandard/files/logo.png][height=\lineheight] \infull{AGL} (AGL) \footnote{\goto{automotivelinux.org}[url(https://www.automotivelinux.org/)]}
% https://pbs.twimg.com/card_img/842390631129452544/bRsGtEvm?format=png&name=600x314
\sym{»} \externalfigure[http://i.imgur.com/nflsTUw.png][method=png,height=\lineheight] Automotive Cyber Security \footnote{\goto{nccgroup.trust/uk/our-solutions/your-sectors/transport/automotive}[url(https://www.nccgroup.trust/uk/our-solutions/your-sectors/transport/automotive/)]}
\sym{»} \externalfigure[https://autoalliance.org/wp-content/themes/autoalliance/assets/img/auto-alliance.png][height=\lineheight] Alliance of Automobile Manufacturers \footnote{\goto{autoalliance.org/connected-cars/cybersecurity}[url(https://autoalliance.org/connected-cars/cybersecurity/)]}
\stopitemize

\blank[line]

These platforms enable to share research findings and discussing general trends of cyber security relating to the automotive industry. In order to convince the business to change course on machine learning initiatives for cyber threat prevention the advantages have to be communicated and proof-of-concept solutions need to be presented to be able to reached out to the community with solutions that are practical to the industry. \par
Specialised security conferences around cyber security for automotive are: \par

\blank[line]

\startitemize[joinedup,nowhite]
\sym{»} \framed[background=color,backgroundcolor=black]{\externalfigure[https://plsadaptive.s3.amazonaws.com/gmedia/_MFq9eacs_summit.png][height=\lineheight]} Automotive Cyber Security Summit \footnote{\goto{automotivecybersecurity.iqpc.com}[url(https://automotivecybersecurity.iqpc.com/)]}
% https://www.cybersecurecar.com/europe/res/display/default/chrome/identifier-17-europe.svg
\sym{»} \externalfigure[http://i.imgur.com/kmTUAz5.png][height=\lineheight] Cyber Secure Car – The automotive cyber security conference \footnote{\goto{cybersecurecar.com}[url(https://www.cybersecurecar.com/)]}
\sym{»} \externalfigure[http://www.tu-auto.com/cyber-security/images/header/TU-Automotive.png][height=\lineheight] TU-Automotive Cybersecurity USA \footnote{\goto{tu-auto.com/cyber-security}[url(http://www.tu-auto.com/cyber-security/)]}
\stopitemize

\blank[line]

These events present a good chance to engage with the community for the similarities in the requirements and convince them with the findings in this paper. \par
\stopsection

% Total number of words per section: 151/250

\blank[line]

Detailed design descriptions were given in this chapter and the use case taken as an example in the paper was elaborated upon and justified. Closely tied to this is the following chapter, which has taken these design discussions as a base and will give accounts to the results found and the analysis throughout the work that has been done. \par

\stopchapter

% Total number of words: 1958/2000

\stopcomponent
