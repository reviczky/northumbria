\startcomponent design
\product dissertation

\usemodule[chart]
\setupFLOWcharts[nx=4,ny=2,dx=2\bodyfontsize,dy=2\bodyfontsize,maxwidth=\textwidth,height=3\lineheight]
\setupFLOWshapes[framecolor=black,background=color,backgroundcolor=white]

\setuphead[chapter][textstyle={\tfd\sc}]
\startchapter[title=Design,reference=chapter:design]

% This should discuss any practical work done.
% If a significant software product is built then this should include a discussion of each stage of the software development life cycle, including:
% Definition of requirements; Design; Implementation; Testing
% Along with consideration of any tools and technologies used (languages, IDEs, third-party components used, etc).
% Similarly if your practical work has consisted of experimental work, then this should include a discussion of each stage of the experimental process including
% Definition of experimental variables and other requirements; Design of experiments; Implementation; Collection of Results
% Along with consideration of any tools and technologies used.
% This discussion should outline the choices made, along with any alternatives considered and the reasons for those choices.

% This section should assess the discussion of the practical work you have done, such as requirements analysis, software design, construction, installation, experimental work.
% Your testing and/or data collection approach should be included in this section although the results from this form part of the next section.
% You should make it clear what you have done, but you should also include rationales for the approaches and techniques used, as well as a discussion of any ideas that have been rejected.

Any practical or significant work that has been done is going to be described and discussed in this chapter, which includes the requirements, design and planning phases. \par
A detailed illustration of the experimental case study will be given along the whole design of the project. Looking at the supporting software utilised, the requirements, construction and usage will be explained. Each of the tests will be validated and the data collection for such testing shown. \par

\blank[line]

The main idea of the work is based on the probabilistic Recursive Bayesian estimation (also known as Bayes filter) \footnote{\goto{Wikipedia: wikipedia.org/wiki/Recursive Bayesian estimation}[url(https://en.wikipedia.org/wiki/Recursive_Bayesian_estimation)]} technique which will be referred to in all instances. \par
Looking at the future states, the probability of the current given state in relation of the very previous state is conditionally independent and expressed with the following formula: \par

\placenamedformula[formula:markov]{Markov Assumption}
\definereferenceformat[eqref][left=(,right=)]
%\setupnumber[formula][way=bysection]
\setupformulas[location=left]
\startformula
p(\bf{x}_k|\bf{x}_{k-1},\bf{x}_{k-2},\dots,\bf{x}_0) = p(\bf{x}_k|\bf{x}_{k-1})
\stopformula

Where the probability distribution of the states at the \infull{HMM} (HMM) is written as: \par

\placenamedformula[formula:hmm]{Hidden Markov Model (HMM)}
\startformula
% p(\bf{z}_k|\bf{x}_k,\bf{x}_{k-1},\dots,\bf{x}_{0}) = p(\bf{z}_k|\bf{x}_{k})
p(\bf{x}_0,\dots,\bf{x}_k,\bf{z}_1,\dots,\bf{z}_k) = p(\bf{x}_0)\prod_{i=1}^k p(\bf{z}_i|\bf{x}_i)p(\bf{x}_i|\bf{x}_{i-1})
\stopformula

The following figure illustrates the flow of a Bayesian Network within a HMM: \par

% \startplacefigure[title={Kalman Filter \\ \tfx © Wikipedia},list={Kalman},reference=figure:kalman]
\startplacefigure[title={Hidden Markov Model},list={Hidden Markov Model},reference=figure:markov]
% \externalfigure[https://upload.wikimedia.org/wikipedia/commons/thumb/8/81/HMM_Kalman_Filter_Derivation.svg/466px-HMM_Kalman_Filter_Derivation.svg.png][width=50mm]
\scale[scale=500]{
\startFLOWchart[markov]
\startFLOWcell
 \name{bay0}
 \location{1,1}
 \shape{0}
 \connection[rl]{bay1}
\stopFLOWcell
\startFLOWcell
 \name{bay1}
 \location{2,1}
 \shape{35}
 \text{$$x_{k-1}$$}
 \connection[rl]{bay2}
 \connection[bt]{bay3}
\stopFLOWcell
\startFLOWcell
 \name{bay2}
 \location{3,1}
 \shape{35}
 \text{$$x_k$$}
 \connection[bt]{bay4}
 \connection[rl]{bay99}
\stopFLOWcell
\startFLOWcell
 \name{bay3}
 \location{2,2}
 \shape{24}
 \text{$$z_{k-1}$$}
\stopFLOWcell
\startFLOWcell
 \name{bay4}
 \location{3,2}
 \shape{24}
 \text{$$z_k$$}
\stopFLOWcell
\startFLOWcell
 \name{bay99}
 \location{4,1}
 \shape{0}
\stopFLOWcell
\stopFLOWchart
\FLOWchart[markov]
}
\stopplacefigure

\startsection[title=Experimental Work]
% Definition of experimental variables and other requirements; Design of experiments; Implementation; Collection of Results

% requirements analysis, software design, construction, installation, experimental work
% testing and/or data collection approach
% make it clear what you have done, but you should also include rationales for the approaches and techniques used, as well as a discussion of any ideas that have been rejected

detailed version of experiments \par
Experiments \par
Considered Tools: pymc, stan, pybayes, comp site ; choices, alternatives, reasons for choices made \par
Literature: Automobile blackbox, Air blackbox, Bayes in Security (Darktrace), Tesla \par
Risk: Threat Prevention \par
Threats: Hijacking, Pilot Error \par
Links: \par
"Tesla Model S lässt sich von fern kapern" \goto{Tesla Model S}[url(http://www.heise.de/newsticker/meldung/Tesla-Model-S-laesst-sich-von-fern-kapern-3327510.html)] \\
"A weakly informative default prior distribution for logistic and other regression models" \cite[authoryear][gelman2008] \footnote{\goto{stat.columbia.edu/~gelman/research/published/priors11.pdf}[url(http://www.stat.columbia.edu/~gelman/research/published/priors11.pdf)]} \\
% \goto{CamDavidsonPilon}[url(http://nbviewer.jupyter.org/github/CamDavidsonPilon/Probabilistic-Programming-and-Bayesian-Methods-for-Hackers/blob/master/Chapter6_Priorities/Chapter6.ipynb)] \\
% "Bayesian Methods for Hackers" \cite[authoryear][Davidson-Pilon:2015:BMH:2851115] \footnote{\goto{Getting our priorities straight}[url(http://nbviewer.jupyter.org/github/CamDavidsonPilon/Probabilistic-Programming-and-Bayesian-Methods-for-Hackers/blob/master/Chapter6_Priorities/Ch6_Priors_PyMC3.ipynb)]} \\
"Probabilistic Programming and Bayesian Methods for Hackers" \cite[authoryear][Davidson-Pilon:2015:BMH:2851115] \footnote{\goto{camdavidsonpilon.github.io/Probabilistic-Programming-and-Bayesian-Methods-for-Hackers}[url(http://nbviewer.jupyter.org/github/CamDavidsonPilon/Probabilistic-Programming-and-Bayesian-Methods-for-Hackers/blob/master/Chapter1_Introduction/Ch1_Introduction_PyMC3.ipynb)]} \\
% \goto{Probabilistic}[url(http://camdavidsonpilon.github.io/Probabilistic-Programming-and-Bayesian-Methods-for-Hackers/)] \\
"Bayesian regression with STAN: Part 1 normal regression" \footnote{\goto{r-bloggers.com/bayesian-regression-with-stan-part-1-normal-regression}[url(https://www.r-bloggers.com/bayesian-regression-with-stan-part-1-normal-regression/)]} \\
\startplacefigure[title={Pairwise Correlation},list={Pairwise Correlation},reference=figure:correlation]
 \externalfigure[https://i0.wp.com/datascienceplus.com/wp-content/uploads/2016/01/stan_norm2.png][width=75mm]
\stopplacefigure
"ShinyStan" \footnote{\goto{jsg2201.shinyapps.io/ShinyStanDemo}[url(https://jsg2201.shinyapps.io/ShinyStanDemo/)]} \\
\stopsection

% Total number of words per section: 250

\startsection[title={Data Collection},reference=section:datacollection]

Planning and data collection consisted of two essential parts that needs to be defined before forward processing is possible. Firstly, a collection of sensor readings that could make up the schematics of the open data \footnote{\goto{opendatahandbook.org}[url(http://opendatahandbook.org/)]}. In that regards the following readings have been defined: \par

\startitemize[joinedup,nowhite]
\sym{»} Speed metrics (m/s) through air pressure
\sym{»} Geolocation (longitude and latitude) via GPS positioning
\sym{»} Elevation (altitude) via GPS positioning
\sym{»} Orientation (cardinal direction) via gyroscopes
\sym{»} UTC date-stamps via the \infull{NTP} (NTP)
\sym{»} IPv6 addresses for unique identification of systems (see The Internet Society \footnote{\goto{internetsociety.org/ipv6-frequently-asked-questions}[url(https://www.internetsociety.org/ipv6-frequently-asked-questions\#seventeen)]})
\stopitemize

Secondly, once the data types have been defined, the creation of a database from these metrics have been based on the established format used for broadcasting aeroplanes in the aviation industry via ADB-S, see full \in{Table}[table:ads]. \par
\startplacefigure[title={ADS-B CSV Database},list={ADS-B CSV Database},reference=figure:ads_csv]
% http://cdn-misc.pinkfroot.com/3C6618.csv
\starttyping
adshex,lat,lon,mtime,altitude,heading,speed,vertrate,flightno
3C6618,44.1141,6.32636,1427189963,10275,26.3,409.2,-2560,4U9525
3C6618,44.0445,6.28003,1427189923,12300,26.6,412.8,-3008,4U9525
3C6618,44.0196,6.26355,1427189908,13000,26.1,429.8,-3520,4U9525
3C6618,43.9671,6.22835,1427189879,14700,26.5,428,-3456,4U9525
\stoptyping
\stopplacefigure
Collecting a sample set of data so that the experiments could be analysed for correlation and anomaly detection, two routes have been created (\in{Section}[section:experiments]), based on real route data but artificial motion patterns to avoid ethical conflicts on personal data and safety. The full table is available on github.com \footnote{\goto{github.com/reviczky/northumbria/tree/master/dissertation/database}[url(https://github.com/reviczky/northumbria/tree/master/dissertation/database/)]}. \par
Having established the data sources and the data types it should be noted that the local collection of cached data is important to run on-the-fly data analysis on the big data as well as for redundancy, error correction and not least forensics (for legal reasons). This could be achieved with data recorders, also commonly known as black boxes for connected cars, but a central data collection service needs to be considered as well, especially when thinking about ITS. \par
With the data at hand, the application of modern data science to analyse the data will play a vital role for this project. \par
\stopsection

% Total number of words per section: 274/250

\startsection[title={Estimation Model},reference=section:model]
The aim of the project (\in{Section}[section:aim]) was to chose the right model for predicting future states in order to deduce anomaly behaviour. The literature review (\in{Chapter}[chapter:literature]) highlighted a variety of mathematical models that could be applied on the side of existing models currently used in the transportation industry. With the proposal on settling with Bayesian estimation techniques the focus in this section is on recursive Bayesian estimation (or in short Bayes filter). \par
The following three applications are available for Bayesian estimation on general probabilistic methods: \par

\startitemize[joinedup,nowhite]
\sym{»} Kalman filter
\sym{»} Markov process
\sym{»} \infull{HMM} (HMM), see \in{Formula}{.}[formula:hmm]

\stopitemize

Estimating unknown future states (through incoming metrics) via recursive density functions allow the usage of three types of mathematical models: \par

\startitemize[joinedup,nowhite]
\sym{»} filtering
\sym{»} smoothing
\sym{»} prediction
\stopitemize

These sequential Bayesian filtering methods, which are extensively used in robotics and other embedded control devices are prefect to by applied on the data collected in the previous section (\in{Section}[section:datacollection]) in order to perform prediction analysis. \par
Plotting density functions with these models is to enable comparisons through correlation for behavioural patterns (algorithms) this project is interested in analysing. \par
The following Bayesian estimation equational expressions are used in libraries for the plotting of the use cases in this project: \par

\placenamedformula[formula:markov]{Markov Process}
\startformula
P(x(t_n) <= x_n | x(t_{(n-1)}),\dots,x(t_1)) = P(x(t_n) <= x_n | x(t_{(n-1)}))
\stopformula

% MathWorld Bayesian Entry
A detailed explanation of the formula is available on the reference work MathWorld \footnote{\goto{mathworld.wolfram.com/MarkovProcess.html}[url(http://mathworld.wolfram.com/MarkovProcess.html)]}. \par

Further, looking at the Kalman filter \footnote{\goto{mathworld.wolfram.com/KalmanFilter.html}[url(http://mathworld.wolfram.com/KalmanFilter.html)]} to optimise imprecise linear or near-linear data is expressed in this formula: \par

\placenamedformula[formula:markov2]{Markov Process Stochastic}
\startformula
P(x(t_n) <= x_n | x(t)\ for\ all\ t <= t_{(n-1)}) = P(x(t_n) <= x_n | x(t_{(n-1)}))
\stopformula

Lastly, looking at probabilistic models using Bayesian methodology in programming for a unifying framework \cite[authoryear][diard:hal-00019254] several filtering and smoothing algorithms \cite[authoryear][Srkk:2013:BFS:2534502] will be incorporated into the method used in this project. \par
\stopsection

% Total number of words per section: 260/250

\startsection[title={Correlation Graphs}]
Utilising the formulas in the previous section (\in{Section}[section:model]) \par

\startcolumns[n=2,rule=on]
\startplacefigure[title={Convergence of Expected Values \tfx © Cam Davidson-Pilon},list={Convergence of Expected Values},reference=figure:convergence]
 % http://nbviewer.jupyter.org/github/CamDavidsonPilon/Probabilistic-Programming-and-Bayesian-Methods-for-Hackers/blob/master/Chapter4_TheGreatestTheoremNeverTold/Ch4_LawOfLargeNumbers_PyMC2.ipynb
 \externalfigure[http://i.imgur.com/aGFa1XV.png][width=85mm]
\stopplacefigure
\column
\blank[line] \ 
\startplacefigure[title={Empirical Returns \\ \tfx © Cam Davidson-Pilon},list={Empirical Returns},reference=figure:empirical]
 % http://nbviewer.jupyter.org/github/CamDavidsonPilon/Probabilistic-Programming-and-Bayesian-Methods-for-Hackers/blob/master/Chapter5_LossFunctions/Ch5_LossFunctions_PyMC2.ipynb
 \externalfigure[http://i.imgur.com/Nl1jPMN.png][width=75mm]
\stopplacefigure
\stopcolumns

Profiling, Patterns \par
The following Graphs, Figures and Tables: \par
more in the appendix, see \in{Chart}[chart:rstan] and \in{Chart}[chart:deviation] \par
\stopsection

% Total number of words per section: 250

\startsection[title=Case Study: Connected Cars]
This paper is proving the problem statement and theory on proactive threat detection by the means of a case study with autonomous connected cars. \par
The specific experiments for the two chosen routes are described in the methodology (\in{Section}[section:experiments]). \par
Route 1 is taken with the scenario of a connected car on the "Autobahn" in Germany, whereas the second example is in contrast with the driving side being on the left in the United Kingdom. This way, the results should be not biased or skewed towards patterns that would not be supportive of the theory. \par
Sensors readings were oriented to the ones seen in the semi-autonomous vehicles found in the likes of Tesla and BMW (see \in{Appendix}[appendix:figures] \in{Figure}[figure:tesla]). Sampling on the movements of available routes is also of advantage, as there is a very good set for comparison of machine learning and profile based behavioural analysis. \par
The advantage of an example scenario with connected cars instead of, for example, connected planes or drones is that with vehicles on the road it is also possible to analyse human behaviour in contrast of programmed system movement. \par
The threat modeling on connected cars is represented in this figure: \par

\startplacefigure[title={Connected Car Threat Modeling \\ © Guardtime},list={Threat Modeling},reference=figure:threatmodel]
 \externalfigure[http://f.edicy.com/0000/0036/0235/photos/car_diagram_large.jpg][width=85mm]
\stopplacefigure

Organisation like the Automotive Cyber Security \footnote{\goto{automotivecybersecurity.iqpc.com}[url(https://automotivecybersecurity.iqpc.com/)]} and the Cyber Secure Car \footnote{\goto{cybersecurecar.com}[url(https://www.cybersecurecar.com/europe/)]} enables to engage with the community to advance and present the discussions around cyber threat and resilience findings this paper has presented and can bridge the gap between research and industry. \par
\stopsection

% Total number of words per section: 238/250

\startsection[title=Particular Events]
Particular events \par
Events \par
\stopsection

% Total number of words per section: 250

\startsection[title=Similar Requirements for the Community]
Community for connected cars. \par
similar requirements to convince the community \par
\stopsection

% Total number of words per section: 250

\blank[line]

Detailed design descriptions were given in this chapter and the use case taken as an example in the paper was elaborated upon and justified. Closely tied to this is the following chapter, which has taken these design discussions as a base and will give accounts to the results found and the analysis throughout the work that has been done. \par

\stopchapter

% Total number of words: 218/2000

\stopcomponent
