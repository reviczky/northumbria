\startcomponent design
\product dissertation

\usemodule[chart]
\setupFLOWcharts[nx=4,ny=2,dx=2\bodyfontsize,dy=2\bodyfontsize,maxwidth=\textwidth,height=3\lineheight]
\setupFLOWshapes[framecolor=black,background=color,backgroundcolor=white]

\setuphead[chapter][textstyle={\tfd\sc}]
\startchapter[title=Design,reference=chapter:design]

% This should discuss any practical work done.
% If a significant software product is built then this should include a discussion of each stage of the software development life cycle, including:
% Definition of requirements; Design; Implementation; Testing
% Along with consideration of any tools and technologies used (languages, IDEs, third-party components used, etc).
% Similarly if your practical work has consisted of experimental work, then this should include a discussion of each stage of the experimental process including
% Definition of experimental variables and other requirements; Design of experiments; Implementation; Collection of Results
% Along with consideration of any tools and technologies used.
% This discussion should outline the choices made, along with any alternatives considered and the reasons for those choices.

% This section should assess the discussion of the practical work you have done, such as requirements analysis, software design, construction, installation, experimental work.
% Your testing and/or data collection approach should be included in this section although the results from this form part of the next section.
% You should make it clear what you have done, but you should also include rationales for the approaches and techniques used, as well as a discussion of any ideas that have been rejected.

Any practical or significant work that has been done is going to be described and discussed in this chapter, which includes the requirements, design and planning phases. \par
experiments (case study) \par
This chapter is looking at the design of the project. \par

\blank[line]

Recursive Bayesian estimation:

\placenamedformula[formula:bayesian]{Recursive Bayesian estimation}
\definereferenceformat[eqref][left=(,right=)]
%\setupnumber[formula][way=bysection]
\setupformulas[location=left]
\startformula
p(\bf{x}_k|\bf{x}_{k-1},\bf{x}_{k-2},\dots,\bf{x}_0) = p(\bf{x}_k|\bf{x}_{k-1})
\stopformula

\placenamedformula[formula:bayesian]{Recursive Bayesian filter}
\startformula
p(\bf{z}_k|\bf{x}_k,\bf{x}_{k-1},\dots,\bf{x}_{0}) = p(\bf{z}_k|\bf{x}_{k})
\stopformula

% picture from Bayesian Kalman from Wikipedia
% \startplacefigure[title={Kalman Filter \\ \tfx © Wikipedia},list={Kalman},reference=figure:cps]
\startplacefigure[title={Kalman Filter},list={Kalman},reference=figure:cps]
% \externalfigure[https://upload.wikimedia.org/wikipedia/commons/thumb/8/81/HMM_Kalman_Filter_Derivation.svg/466px-HMM_Kalman_Filter_Derivation.svg.png][width=50mm]
\scale[scale=500]{
\startFLOWchart[kalman]
\startFLOWcell
 \name{bay0}
 \location{1,1}
 \shape{0}
 \connection[rl]{bay1}
\stopFLOWcell
\startFLOWcell
 \name{bay1}
 \location{2,1}
 \shape{35}
 \text{$$x_{k-1}$$}
 \connection[rl]{bay2}
 \connection[bt]{bay3}
\stopFLOWcell
\startFLOWcell
 \name{bay2}
 \location{3,1}
 \shape{35}
 \text{$$x_k$$}
 \connection[bt]{bay4}
 \connection[rl]{bay99}
\stopFLOWcell
\startFLOWcell
 \name{bay3}
 \location{2,2}
 \shape{24}
 \text{$$z_{k-1}$$}
\stopFLOWcell
\startFLOWcell
 \name{bay4}
 \location{3,2}
 \shape{24}
 \text{$$z_k$$}
\stopFLOWcell
\startFLOWcell
 \name{bay99}
 \location{4,1}
 \shape{0}
\stopFLOWcell
\stopFLOWchart
\FLOWchart[kalman]
}
\stopplacefigure
https://en.wikipedia.org/wiki/Recursive_Bayesian_estimation \par

This formula, see: \in{Equation}{.}[formula:bayesian]. \par

\startsection[title=Experimental Work]
% Definition of experimental variables and other requirements; Design of experiments; Implementation; Collection of Results

detailed version of experiments \par
Experiments \par
Tools: pymc, stan, pybayes, comp site \par
Literature: Automobile blackbox, Air blackbox, Bayes in Security (Darktrace), Tesla \par
Risk: Threat Prevention \par
Threats: Hijacking, Pilot Error \par
Links: \par
\goto{Tesla Model S}[url(http://www.heise.de/newsticker/meldung/Tesla-Model-S-laesst-sich-von-fern-kapern-3327510.html)] \\
\goto{priors11}[url(http://www.stat.columbia.edu/~gelman/research/published/priors11.pdf)] \\
\goto{CamDavidsonPilon}[url(http://nbviewer.jupyter.org/github/CamDavidsonPilon/Probabilistic-Programming-and-Bayesian-Methods-for-Hackers/blob/master/Chapter6_Priorities/Chapter6.ipynb)] \\
\goto{Probabilistic}[url(http://camdavidsonpilon.github.io/Probabilistic-Programming-and-Bayesian-Methods-for-Hackers/)] \\
\goto{bayesian-regression}[url(https://www.r-bloggers.com/bayesian-regression-with-stan-part-1-normal-regression/)] \\
\goto{ShinyStanDemo}[url(https://jsg2201.shinyapps.io/ShinyStanDemo/)] \\
\stopsection

% Total number of words per section: 250

\startsection[title={Data Collection}]
Planning \par
Air blaxbox / car blackbox \par
data science \par
\stopsection

% Total number of words per section: 250

\startsection[title={Estimation Model},reference=section:model]
We will look at three models for Bayesian estimation: Kalman filter, Markov process. \par

Density \par
Model \par
Behavioural algorithm \par
looking into a model \par

% MathWorld Bayesian Entry
http://mathworld.wolfram.com/KalmanFilter.html \par
http://mathworld.wolfram.com/MarkovProcess.html \par

\placenamedformula[formula:markov]{Markov process}
\startformula
P(x(t_n) <= x_n | x(t_{(n-1)}),\dots,x(t_1)) = P(x(t_n) <= x_n | x(t_{(n-1)}))
\stopformula
\placenamedformula[formula:markov2]{Markov process 2}
\startformula
P(x(t_n) <= x_n | x(t)\ for\ all\ t <= t_{(n-1)}) = P(x(t_n) <= x_n | x(t_{(n-1)}))
\stopformula
\stopsection

% Total number of words per section: 250

\startsection[title={Correlation Graphs}]
The following Graphs, Figures and Tables: \par
\stopsection

% Total number of words per section: 250

\startsection[title=Case Study: Connected Cars]
Case study: connected cars. \par
\stopsection

% Total number of words per section: 250

\startsection[title=Particular Events]
Particular events \par
Events \par
\stopsection

% Total number of words per section: 250

\startsection[title=Similar Requirements for the Community]
Community for connected cars. \par
similar requirements to convince the community \par
\stopsection

% Total number of words per section: 250

\blank[line]

In this chapter we have looked at the design of the project. The next chapter is the results collected and analysis. \par

\stopchapter

% Total number of words: 2000

\stopcomponent
