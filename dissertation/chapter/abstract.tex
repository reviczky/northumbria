\startcomponent abstract
\product dissertation

\setuphead[chapter][textstyle={\tfd\sc},commandbefore={\blank[3*halfline]}]
\startchapter[title=Abstract]

\definelayer[abstract]
\setlayer[abstract][x=0mm,y=-53mm]{\framed[frame=off,width=\textwidth,align=middle]{\sc \studentname \blank[line] Proactive threat detection of cyber-physical systems using Bayesian estimation: connected cars as a case study}}
\flushlayer[abstract]

% The abstract is usually about 200 words long and not exceeding 300 words and summarises the motivation for the project, the work done, and the key results.
\startframed[frame=off,topframe=on,bottomframe=on,width=\textwidth,align=width,offset=2mm,rulethickness=2pt,framecolor=darkolivegreen]
With the upcoming disruptive technologies around autonomous driving and cyber-physical systems the increase of cyber threats against such systems is not matched with appropriate security design and incorporation of proactive preventative measures. \par
The aim of this project is to show a novel method of analysing comparable travel routes in real-time to predict anomalies through a use-case of hijacked connected cars. \par
Multiple experiments have been conducted by incorporating different Bayesian estimation techniques to analyse and predict future states based on previous behaviour in order to show an increased window for reaction when encountering anomalies compared with existing machine learning models. \par
It has been shown that detecting real-time deviations for malicious intent with predictive and behavioural methods are far superior than the retrospective comparison of good behaviour and a significantly quicker reaction can be taken to counter cyber threats. \par
By creating behavioural based algorithms to detect anomalies with the use-case of hijacked connected cars the project achieved to demonstrate how to deal with cyber threats by design as early as they happen. \par
A possible proactive warning mechanism could be developed from these findings. \par
% 181 words
%\blank[line]
%\startitemize[joindup,nowhite]
%\item Motivation for the Project
%\item Research Hypothesis/Question/Aim
%\item Work done
%\item Results
%\item Conclusions
%\stopitemize
\stopframed
\stopchapter
\setuphead[chapter][textstyle={\tfd\sc},commandbefore=]

\stopcomponent
