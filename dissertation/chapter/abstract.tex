\startcomponent abstract
\product dissertation

\setuphead[chapter][textstyle={\tfd\sc},commandbefore={\blank[3*halfline]}]
\startchapter[title=Abstract]

\definelayer[abstract]
\setlayer[abstract][x=0mm,y=-53mm]{\framed[frame=off,width=\textwidth,align=middle]{\sc \studentname \blank[line] Proactive threat detection of cyber-physical systems using Bayesian estimation: connected cars as a case study}}
\flushlayer[abstract]

% The abstract is usually about 200 words long and not exceeding 300 words

\startframed[frame=off,topframe=on,bottomframe=on,width=\textwidth,align=width,offset=2mm,rulethickness=2pt,framecolor=darkolivegreen]
% Motivation for the Project
With the upcoming disruptive technologies around autonomous driving of cyber-physical systems, the increase of cyber threats against such systems has not yet been matched with appropriate security by design and lacks approaches to incorporate proactive preventative measures. \par
% Research Hypothesis / Question / Aim
Thus, the aim of this project was to illustrate a novel method of analysing comparable travel routes in real-time to predict anomalies through a use-case of hijacked connected cars. \par
% Work Done
To prove the advancement and benefits with this new approach of predictive modelling in comparison to existing industry-standard signature based machine learning models, multiple simulations have been conducted. This has been done by incorporating different Bayesian estimation techniques to analyse and predict future states based on previous behaviour in order to show a vastly increased time-window for reaction when encountering anomalies. \par
% Key Results
This research paper showed and concluded that detecting real-time deviations for malicious intent with predictive and behavioural methods are far superior in precision and effectiveness than the retrospective comparison of known-good behaviour. Therefore, significantly quicker action can be taken to counter cyber-attacks. \par
% Conclusions
By creating profiling based behavioural algorithms to detect anomalies with the use-case of hijacked connected cars, the project demonstrated how to deal with cyber threats by design as early as they occur in the kill chain. \par
% Further Work
Taking these findings into further research, the creation of a proactive warning mechanism and a reactive engagement or interception of command and control could be developed. \par
\stopframed

%\framedtext[frame=off,offset=0mm,strut=no,width=\textwidth,align=left,style=\ssxx]{\startcolor[darkolivegreen] 232 words \stopcolor}
\stopchapter
\setuphead[chapter][textstyle={\tfd\sc},commandbefore=]

\stopcomponent
