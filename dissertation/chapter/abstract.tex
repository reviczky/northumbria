\startcomponent abstract
\product dissertation

\setuphead[chapter][textstyle={\tfd\sc},commandbefore={\blank[3*halfline]}]
\startchapter[title=Abstract]

\definelayer[abstract]
\setlayer[abstract][x=0mm,y=-53mm]{\framed[frame=off,width=\textwidth,align=middle]{\sc \studentname \blank[line] Proactive threat detection of cyber-physical systems using Bayesian estimation: connected cars as a case study}}
\flushlayer[abstract]

% The abstract is usually about 200 words long and not exceeding 300 words and summarises the motivation for the project, the work done, and the key results.
\startframed[frame=off,topframe=on,bottomframe=on,width=\textwidth,align=width,offset=2mm,rulethickness=2pt,framecolor=darkolivegreen]
With the upcoming disruptive technologies around autonomous driving of cyber-physical systems the increase of cyber threats against such systems has not yet been matched with appropriate security design and the incorporation of proactive preventative measures. \par
Thus, the aim of this project was to present a novel method of analysing comparable travel routes in real-time to predict anomalies through a use-case of hijacked connected cars. \par
To prove the advancement and benefits with this new approach in comparison to existing industry-standard machine learning models, multiple experiments have been conducted by incorporating different Bayesian estimation techniques to analyse and predict future states based on previous behaviour in order to show a vastly increased time-window for reaction when encountering anomalies. \par
This research paper showed and concluded that detecting real-time deviations for malicious intent with predictive and behavioural methods are far superior in precision and effectiveness than the retrospective comparison of good behaviour and therefore, a significantly quicker reaction can be undertaken to counter cyber attacks. \par
By creating profiling based behavioural algorithms to detect anomalies with the use-case of hijacked connected cars the project achieved to demonstrate how to deal with cyber threats by design as early as they happen in the kill chain. \par
Taking these findings into further research, the creation of a proactive warning mechanism and an reactive engagement or interception of command and control could be developed. \par
% 225 words
%\blank[line]
%\startitemize[joindup,nowhite]
%\item Motivation for the Project
%\item Research Hypothesis/Question/Aim
%\item Work done
%\item Results
%\item Conclusions
%\stopitemize
\stopframed
\stopchapter
\setuphead[chapter][textstyle={\tfd\sc},commandbefore=]

\stopcomponent
