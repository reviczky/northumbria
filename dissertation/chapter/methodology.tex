\startcomponent methodology
\product dissertation

\setuphead[chapter][textstyle={\tfd\sc}]
\startchapter[title=Methodology,reference=chapter:methodology]
% All chapters should be topped and tailed.
% In other words, the first paragraph of a chapter should say what the chapter is about;
% the last paragraph should summarise what the chapter was about.

In this chapter it is going to be explained what methods have been used and how the research was being carried out. It gives an overview of the research and its design together with how the data was analysed. A comprehensive list of experiments that have been conducted will conclude this chapter. \par

\blank[line]

For this project, a quantitative data type research methodology approach was chosen underpinned by empirically observed data (sample location data) to support and illustrate experiments on mathematical modelling with the introduction of a new formula to correlate routes. \par

\startsection[title=Research Overview]
% Explain how the research has been carried out
% Where is the data come from
% Method and Sources, Aims for the Methodology, How/Why
% How did it serve the purpose (mora appropriate than other methods)
The research is based on a sample set of route data that has been created in order to show different possible vehicle movement behaviours. The data uses real \infull{GPS} (GPS) coordinates but the routes should only resemble motion patterns and are not taken from real events (see: \in{Table}[table:ads]). This set of data is then analysed and compared with current machine learning techniques and a second time with the proposed new algorithms of behaviour analysis (see: \in{Chart}[chart:rstan]). \par
With the output of the correlation data, metrics of deviation from the norm and time to detect will form the main evidence of significant improvements to detect anomalies. \par
To avoid problems relating with data privacy and \infull{PII} (PII), statistical sampling provides the best method to conduct correlation techniques based on real routes but fictitious behaviour.
Once the proposed technique for profiling and behavioural analysis is proven to be a new and better way of defining anomalies, it can then be run through a database of real recorded routes and behaviours to support and validate the findings in a scenario taken from live data. \par
\stopsection

% Total number of words per section: 183/200

\startsection[title=Research Design]
% Data gathering techniques:
% data optained, data collection, primary/secondary sources
% types of data used: numerical, what do they record
% what does the data mean in context of the research

Data gathering techniques for this research are based on numerical data collection that was created through capturing geolocation available through secondary sources in a way that the latitude, longitude and elevation coordinates \footnote{\goto{ordnancesurvey.co.uk/docs/support/guide-coordinate-systems-great-britain.pdf}[url(https://www.ordnancesurvey.co.uk/docs/support/guide-coordinate-systems-great-britain.pdf)]} are real but not specific to any individual's travel route behaviour. The sample size is capped with an approximate target of 100 comparable routes of various vehicle movements. \par
To illustrate the data-set, the following excerpt shows the data types and data fields of the data obtained. For the full table and database defined with open data formats that are being analysed, see \in{ADS Table}[table:ads] in the Appendix. Each type of data is chosen in line with \infull{SI} (SI) measurements and universal or atomic definitions to abstract the proposed method in order to be applicable to a wide range of use-cases. \par
Obtained data includes: System Identity, Universal Date & Time, Global Position, Global Orientation, Ground Speed and Reporting Method. \par

\blank[line]

\placetable[here][table:dataset]{Sample data-set extract}{
\startxtable[foregroundstyle=\ssxx]
\startxtablehead[background=color,backgroundcolor=gray]
\startxrow
\startxcell[foregroundstyle=\ssxx\bf] Cyber-Physical \\ System (ID) \stopxcell
\startxcell[foregroundstyle=\ssxx\bf] Date \\ (UTC) \stopxcell
\startxcell[foregroundstyle=\ssxx\bf] Time \\ (UTC) \stopxcell
\startxcell[foregroundstyle=\ssxx\bf] Position \\ (latitude) \stopxcell
\startxcell[foregroundstyle=\ssxx\bf] Position \\ (longitude) \stopxcell
\startxcell[foregroundstyle=\ssxx\bf] Position \\ (elevation) \stopxcell
\startxcell[foregroundstyle=\ssxx\bf] Orientation \\ (degree) \stopxcell
\startxcell[foregroundstyle=\ssxx\bf] Orientation \\ (cardinal) \stopxcell
\startxcell[foregroundstyle=\ssxx\bf] Ground Speed \\ (m/s) \stopxcell
\startxcell[foregroundstyle=\ssxx\bf] Reporting \\ (system) \stopxcell
\stopxrow
\stopxtablehead
\startxtablebody[align=middle]
\startxrow
% \startxcell 2017:0db8:85a3:0000:0000:8a2e:0370:7334 \stopxcell
\startxcell 2017:0db8:85a3 \stopxcell
\startxcell 2017-03-11 \stopxcell
\startxcell 06:50:27 \stopxcell
\startxcell 47.82520828 \stopxcell
\startxcell 12.54956887 \stopxcell
\startxcell 456 \stopxcell
\startxcell 190 \stopxcell
\startxcell \WORD{S} \stopxcell
\startxcell 9 \stopxcell
\startxcell ADS-B \stopxcell
\stopxrow
\startxrow
\startxcell 2017:0db8:85a3 \stopxcell
\startxcell 2017-03-11 \stopxcell
\startxcell 06:50:29 \stopxcell
\startxcell 48.24533500 \stopxcell
\startxcell 12.53889999 \stopxcell
\startxcell 455 \stopxcell
\startxcell 191 \stopxcell
\startxcell \WORD{S} \stopxcell
\startxcell 11 \stopxcell
\startxcell ADS-B \stopxcell
\stopxrow
\startxrow
\startxcell[nx=10] \vdots \stopxcell
\stopxrow
\startxrow
\startxcell 2017:0db8:85a3 \stopxcell
\startxcell 2017-03-11 \stopxcell
\startxcell 15:30:40 \stopxcell
\startxcell 48.24545299 \stopxcell
\startxcell 12.53923599 \stopxcell
\startxcell 449 \stopxcell
\startxcell 12 \stopxcell
\startxcell \WORD{N} \stopxcell
\startxcell 7 \stopxcell
\startxcell ADS-B \stopxcell
\stopxrow
\stopxtablebody
\stopxtable
}

% http://031c074.netsolhost.com/WordPress/wp-content/uploads/2016/05/MH804-FlightAware.jpg

Many more metrics could be captured and incorporated into the data-set, but this minimalistic set of data will suffice for proving the theory of profiling and behavioural analysis this paper sets out. \par
\stopsection

% Total number of words per section: 185/200

\startsection[title=Data Analysis]
% https://www.postgrad.com/advice/exams/dissertations_and_theses/dissertation_methodology/
% https://www.postgrad.com/advice/phd/research_methods/quantitative_research/

Having the underlying raw data of comparable routes as a base for assembling and quantitative data analysis the following process is used to interpret the data. \par
The proposed algorithm to analyse the data is based on Recursive Bayesian estimation techniques (Bayes filters), more specifically on Sequential Bayesian filtering (including: filtering, smoothing and prediction). For this project the focus is on Bayesian filtering for the estimation of future states within the route. \par
Free and open source tools have been acquired to help with the statistical modelling of probability and density functions and the plotting of the experiments \footnote{\goto{github.com/CamDavidsonPilon/Probabilistic-Programming-and-Bayesian-Methods-for-Hackers}[url(https://github.com/CamDavidsonPilon/Probabilistic-Programming-and-Bayesian-Methods-for-Hackers)]}. The core tools that have been utilised include: \par

\startitemize[joinedup,nowhite]
\item Stan full Bayesian statistical inference with MCMC sampling (including the interfaces for R and Python): Available at \goto{mc-stan.org}[url(http://mc-stan.org/)]
\item PyMC for Bayesian statistical modelling and Probabilistic Machine Learning (version 2 and 3): Available at \goto{pystan.readthedocs.io}[url(https://pystan.readthedocs.io/en/latest/)], \goto{github.com/markdregan/Bayesian-Modelling-in-Python}[url(https://github.com/markdregan/Bayesian-Modelling-in-Python)]
\stopitemize

% http://twiecki.github.io/blog/2013/09/12/bayesian-glms-1/
% http://andrewgelman.com/2015/10/15/whats-the-one-thing-you-have-to-know-about-pystan-and-pymc-click-here-to-find-out/

More tools have been used in the development of the experiments but are not forming part of the research: \par

\startitemize[joinedup,nowhite]
\item PyBayes for recursive Bayesian estimation (Bayesian filtering): Available at \goto{github.com/strohel/PyBayes}[url(https://github.com/strohel/PyBayes)]
\item BEST Bayesian estimation for two groups (with the interface for R): Available at \goto{github.com/strawlab/best}[url(https://github.com/strawlab/best)], \goto{github.com/mvuorre/bestan}[url(https://github.com/mvuorre/bestan)]
\stopitemize

% theory of interpreting the data
Results of the statistical modelling will be interpreted with metrics of deviations of optimal routes defined with Bayesian filters calculated in percentage as key indicators. The balance of how much deviation to allow until it is flagged as anomaly is the biggest challenge. \par
% results conclusive
As the deviation is based on a comparison of density of comparable profiles of behaviour, the results will be conclusive for any significant deviation.
% any factors in the results
False-positives and bad data is factored into the algorithm and would statistically be not significant. \par

The exact same data is also used to conduct the comparison with current signature based methods. \par
\stopsection

% Total number of words per section: 281/200

\startsection[title=Experiments to Consider]
The proof of the research question is based on specific experiments of hijacked connected cars. \par
Hijacking is defined as an attack from the cyberspace into the connected vehicle and changing the pre-defined route to a different destination, but could also be a more malicious intent of deliberate harm of driving into the wrong direction or off-street. \par

\blank[line]

On a holistic level a general experiment is proposed and formed as the following sequence of events: \par
\startitemize[joinedup,nowhite]
\item Defining a specific route between two cities (from A to B) - this would be calculated and followed by a navigation system
\item Sampling of different profiles along the route for different behaviours
\item Specification of the cyber-attack and route change being carried out (hijacking)
\item Resulting in detection of the deviation from the norm route with Bayesian filtering (warning)
\stopitemize

\blank[line]

The following specific experiments have been conducted and presented in this paper. \par

{\bf Experiment 1:} \\
Sample route is from Munich, Germany to Frankfurt, Germany on the motorways. Vehicles include connected sedan cars and connected \infull{HGV} (HGV) trucks. The hijacking of a connected sedan to deviate the car to Stuttgart will be programmed to the system. The deviation of more than 5\% of comparable routes with Bayesian filtering will trigger a warning. \par
{\bf Experiment 2:} \\
Sample route is from London, United Kingdom to Birmingham, United Kingdom on normal roads and bridges. A mix of connected cars with one car being hacked and sent into the wrong direction but staying on the route. The anomaly through Bayesian estimation will recognise the direction being abnormal and will trigger a warning. \par

\blank[line]

Each experiment will be processed with the tools for applying the Bayesian filtering and plotted with the data points in the route giving the median line of deviation. \par
\stopsection

% Total number of words per section: 285/300

\blank[line]

In summary, this chapter explained the given methodology that has been used and applied for conducting quantitative data analysis of specific experiments in this project in order to prove the research question. The next step is the literature review followed by the sections of the design of the project. \par

\stopchapter

% Total number of words: 1073/1000

\stopcomponent
