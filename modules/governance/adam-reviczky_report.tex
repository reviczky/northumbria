% engine=luajittex

\setupinteraction[state=start]

\setuppapersize[A4,portrait][A4,portrait]

\setuplayout
    [
        backspace=10mm,
        cutspace=10mm,
        width=190mm,
        topspace=10mm,
        header=10mm,
        footer=10mm,
        height=277mm,
    ]

\define\studentname{Adam Reviczky}
\define\studentnumber{14044980}
\define\modulename{Information Governance and Security}
\define\documentname{Report}

\startsetups[headertext]
    \startframed[frame=off,align=middle,width=fit]\bf\WORD
        Student Name: {\studentname} {\|} Student Number: {\studentnumber} \\
        Module: {\modulename} ({\documentname})
    \stopframed
\stopsetups

\setupheadertexts[\directsetup{headertext}]

\setuphead[section][style=\bfb]
\setuphead[subject][style=\tf]

\usemodule[filter]

\defineexternalfilter
    [pandoc]
    [
        filter={pandoc --no-wrap
            -f \externalfilterparameter{format} -t context
            -o \externalfilteroutputfile},
        format=markdown,
    ]

\setuppagenumbering[location=footer,left={\endash\ },right={\ \endash}]
\setupbodyfont[pagella,10pt]

\setupinteraction[state=start]
\placebookmarks[chapter,section,subsection][chapter,section]
\setupinteractionscreen[option=bookmark]

\setupinteraction
    [
        state=start,
        title={Information Governance and Security},
        author={Adam Reviczky},
        subtitle={Report},
        keyword={northumbria},
    ]

\setupspellchecking[state=start,method=2]
\ctxlua{languages.words.threshold=1}

\setupbibtex[database={report},sort=author]

\showframe

\starttext
    \processpandocfile{report.md}

    \startsection[title={Format}]
    \startsubject[title={Word count}]
    \startcolor[black] \tfx
        Total number of words (with a theshold of 1 character): {\bf \ctxlua{local data = dofile"\jobname.words"; context(data.total)}} \\
        Maximum number of words allowed: {\bf 2,500}
    \stopcolor
    \stopsubject
    \startsubject[title={Citation}]
    \startcolor[black] \tfx
        References: {\bf American Psychological Association (APA) style} \\
        Bibiliography: {\bf American Psychological Association (APA) style}
    \stopcolor
    \stopsubject
    \startsubject[title={Layout}]
    \startcolor[black] \tfx
        Paper Size: {\bf A4, Portrait} \\
        Font Family: {\bf TeX Gyre Pagella, 10pt} \\
        Margins: \\
        \startxtable
            \startxrow \startxcellgroup[background=color,backgroundcolor=lightgray,foregroundcolor=black] \startxcell \WORD{Setup} \stopxcell \startxcell \WORD{Dimension} \stopxcell \startxcell \WORD{Description} \stopxcell \stopxcellgroup \stopxrow
            \startxrow \startxcell Backspace \stopxcell \startxcell[align={middle}] {\bf \startluacode local layout = tex.dimen.backspace context("%s mm", number.tomillimeters(layout, "%.0f")) \stopluacode} \stopxcell \startxcell Space from the left rim of the paper to the left rim of the main text area. \stopxcell \stopxrow
            \startxrow \startxcell Cutspace \stopxcell \startxcell[align={middle}] {\bf \startluacode local layout = tex.dimen.cutspace context("%s mm", number.tomillimeters(layout, "%.0f")) \stopluacode} \stopxcell \startxcell Space from the right rim of the paper to the right rim of the main text area. \stopxcell \stopxrow
            \startxrow \startxcell Topspace \stopxcell \startxcell[align={middle}] {\bf \startluacode local layout = tex.dimen.topspace context("%s mm", number.tomillimeters(layout, "%.0f")) \stopluacode} \stopxcell \startxcell Space above the header from the top rim of the paper to the top rim of the header. \stopxcell \stopxrow
            \startxrow \startxcell Header \stopxcell \startxcell[align={middle}] {\bf \startluacode local layout = tex.dimen.headerheight context("%s mm", number.tomillimeters(layout, "%.0f")) \stopluacode} \stopxcell \startxcell The height of the header area. \stopxcell \stopxrow
            \startxrow \startxcell Footer \stopxcell \startxcell[align={middle}] {\bf \startluacode local layout = tex.dimen.footerheight context("%s mm", number.tomillimeters(layout, "%.0f")) \stopluacode} \stopxcell \startxcell The height of the footer area. \stopxcell \stopxrow
        \stopxtable \\
    \stopcolor
    \stopsubject
    \stopsection
\stoptext
